\documentclass[12pt]{book}
\parindent=0px
\usepackage{amssymb, amsmath, fullpage, graphicx}
\usepackage{enumitem, siunitx, tikz}
\usepackage[utf8]{inputenc}
\usepackage[T1]{fontenc}

\newcommand{\set}[1]{\{#1\}}
\newcommand{\Set}{\text}

\newcommand{\defn}[1]{\textbf{#1}}
\newcommand{\solution}[2]{\item #1\\ \textbf{Solution}: #2}
\newcommand{\thm}[1]{\underline{\textsc{#1}}}

\newcommand{\qed}{\blacksquare}
\newcommand{\comp}{\overline}
\newcommand{\env}[2]{\begin{#1} #2 \end{#1}}
\newcommand{\tuple}[1]{\langle#1\rangle}

\newcommand{\then}{\Rightarrow}
\renewcommand{\iff}{\Leftrightarrow}
\newcommand{\goesto}{\rightarrow}

\newcommand{\floor}[1]{\lfloor#1\rfloor}
\newcommand{\ceil}[1]{\lceil#1\rceil}

% sets %
\newcommand{\NAT}{\mathbb{N}}
\newcommand{\INT}{\mathbb{Z}}
\newcommand{\RAT}{\mathbb{Q}}
\newcommand{\REAL}{\mathbb{R}}
\newcommand{\COMP}{\mathbb{C}}

% function families from function to boolean %
\newcommand{\Inj}[1]{\text{Inj}\left(#1\right)}
\newcommand{\Surj}[1]{\text{Surj}\left(#1\right)}
\newcommand{\Bij}[1]{\text{Bij}\left(#1\right)}



\begin{document}
\chapter{Sets and Relations}

\renewcommand{\labelenumi}{2.\arabic{enumi}}
\begin{enumerate}
\solution{Explain why $2 \in \set{1,2,3}$.}{$2 \in \set{1,2,3}$ by definition: 2 is an element of $\set{1,2,3}$.}
\solution{Is $\set{1,2} \in \set{\set{1,2,3}, \set{1,3},1,2}$? Justify your answer.}{$\set{1,2} \notin \set{\set{1,2,3},\set{1,3},1,2}$ since it does not appear as a member of the given set.}
\solution{Try to devise a set which is a member of itself.}{The set of all sets.}
\solution{Give an example of sets $A, B,$ and $C$ such that $A \in B$, $B \in C$, and $A \notin C$.}{$\Set{A} = \set{1}$. $\Set{B} = \set{1, \set{1}}$. $\Set{C} = \set{1, \set{1, \set{1}}}$. Then $\Set{A} \in \Set{B}$, $\Set{B} \in \Set{C}$, but $\Set{A} \notin \Set{C}$.}
\solution{Describe in prose each of the following sets.} 
	{\begin{enumerate}
	\solution{$\set{x \in \mathbb{Z} \mid x \text{ is divisible by } 2 \text{ and } x \text{ is divisible by } 3}$}{All integer multiples of 6.}
	\solution{$\set{x \mid x \in A \text{ and } x \in B}$}{All elements common to sets A and B.}
	\solution{$\set{x \mid x \in A \text{ or } x \in B}$}{All elements from set A and from set B.}
	\solution{$\set{x \in \mathbb{Z}^+ \mid x \in \set{x \in \mathbb{Z} \mid \text{for some integer } y,\, x = 2y} \text{ and } x \in \set{x \in \mathbb{Z} \mid \text{for some integer } y,\, x = 3y}}$}{All positive integer multiples of 6.}
	\solution{$\set{x^2 \mid x \text{ is a prime}}$}{The squares of all prime numbers.}
	\solution{$\set{a/b \in \mathbb{Q} \mid a + b = 1 \text{ and } a, b \in \mathbb{Q}}$}{All ratios of rational numbers whose numerator and denominator sum to 1; i.e. all rational numbers.}
	\solution{$\set{(x,y) \in \mathbb{R}^2 \mid x^2 + y^2 = 1}$}{All points on the unit circle.}
	\solution{$\set{(x,y) \in \mathbb{R}^2 \mid y = 2x \text{ and } y = 3x}$}{The single point (0,0).}
	\end{enumerate}}
\solution{Prove that if $a,b,c$, and $d$ are any objects, not necessarily distinct from one another, then $\set{\set{a},\set{a,b}} = \set{\set{c},\set{c,d}}$ iff both $a = c$ and $b = d$.}
{($\Rightarrow$) Let $\set{\set{a},\set{a,b}} = \set{\set{c},\set{c,d}}$. Then $\set{\set{a},\set{a,b}} \subseteq \set{\set{c},\set{c,d}}$ and so in particular $\set{a} \in \set{\set{c},\set{c,d}}$. Suppose $c = d$. Then $\set{\set{c},\set{c,d}} = \set{\set{c},\set{c,c}} = \set{\set{c},\set{c}} = \set{\set{c}}$ and therefore $\set{a} \in \set{\set{c}} \Rightarrow \set{a} = \set{c} \Rightarrow \set{a} \subseteq \set{c} \Rightarrow a \in \set{c} \Rightarrow a = c$. Then since $\set{\set{a},\set{a,b}} \subseteq \set{\set{c},\set{c,d}} = \set{\set{c}} = \set{\set{a}}$ we also have that $\set{a,b} \in \set{\set{a}} \Rightarrow \set{a,b} = \set{a} \Rightarrow \set{a,b} \subseteq \set{a} \Rightarrow b \in \set{a} \Rightarrow a = b$. Now suppose $c \neq d$. Since $\set{a} \in \set{\set{c},\set{c,d}} \Rightarrow \set{a} = \set{c} \Rightarrow a = c$. By $\set{\set{a},\set{a,b}} \subseteq \set{\set{c},\set{c,d}}$ we also have that $\set{a,b} = \set{c,b} \in \set{\set{c},\set{c,d}}$ and so $\set{c,b} = \set{c,d} \Rightarrow b = d$.\\ ($\Leftarrow$) Let $a = c$ and $b = d$. Then $\set{\set{a},\set{a,b}} = \set{\set{c},\set{c,d}}$. $\qed$}
\end{enumerate}

\hrulefill

\renewcommand{\labelenumi}{3.\arabic{enumi}}
\begin{enumerate}
\item Prove each of the following, using any properties of numbers that may be needed.
	\begin{enumerate}
	\solution{$\set{x \in \mathbb{Z} \mid \text{for an integer } y,\, x = 6y} = \set{x \in \mathbb{Z} \mid \text{for integers } u \text{ and } v,\, x = 2u \text{ and } x = 3v}$.}
	{Let $a \in \set{x \in \mathbb{Z} \mid \text{for an integer } y,\, x = 6y}$. Then $a = 6b$ for an integer $b$. Then $a = 2\cdot 3b$ and $a = 3 \cdot 2b$. Since $b$ is an integer, $3b$ and $2b$ are also integers. Therefore $a \in \set{x \in \mathbb{Z} \mid \text{for integers } u \text{ and } v,\, x = 2u \text{ and } x = 3v}$ and so $\set{x \in \mathbb{Z} \mid \text{for an integer } y,\, x = 6y} \subseteq \set{x \in \mathbb{Z} \mid \text{for integers } u \text{ and } v,\, x = 2u \text{ and } x = 3v}$.\\ Now let $a \in \set{x \in \mathbb{Z} \mid \text{for integers } u \text{ and } v,\, x = 2u \text{ and } x = 3v}$. Then there is an $m$ and $n$ such that $a = 2m$ and $a = 3n$. Then $m - n = \frac{a}{6}$ and so $a = 6(m - n)$. Since $m$ and $n$ are integers, $m - n$ is also an integer and so $a \in \set{x \in \mathbb{Z} \mid \text{for an integer } y,\, x = 6y}$. Therefore $\set{x \in \mathbb{Z} \mid \text{for integers } u \text{ and } v,\, x = 2u \text{ and } x = 3v} \subseteq \set{x \in \mathbb{Z} \mid \text{for an integer } y,\, x = 6y}$.\\ Thus $\set{x \in \mathbb{Z} \mid \text{for an integer } y,\, x = 6y} = \set{x \in \mathbb{Z} \mid \text{for integers } u \text{ and } v,\, x = 2u \text{ and } x = 3v}$. $\qed$}
	\solution{$\set{x \in \mathbb{R} \mid \text{for a real number } y,\, x = y^2} = \set{x \in \mathbb{R} \mid x \geq 0}$}
	{Let $a \in \set{x \in \mathbb{R} \mid \text{for a real number } y,\, x = y^2}$. Then $a = b^2$ for some $b \in \mathbb{R}$. If $b = 0$ then $a = 0$. Otherwise, $a > 0$ since the square of a nonzero real number is positive. Thus $a \in \set{x \in \mathbb{R} \mid x \geq 0}$ and therefore $\set{x \in \mathbb{R} \mid \text{for a real number } y,\, x = y^2} \subseteq \set{x \in \mathbb{R} \mid x \geq 0}$.\\ Now let $a \in \set{x \in \mathbb{R} \mid x \geq 0}$. Then $a \geq 0$. Then we may find a $b \in \mathbb{R}$ such that $a = b^2$: simply pick $b = \sqrt{a} \in \mathbb{R}$. (The square root of any nonnegative real number is again a real number.) Then $a \in \set{x \in \mathbb{R} \mid \text{for a real number } y,\, x = y^2}$ and therefore $\set{x \in \mathbb{R} \mid x \geq 0} \subseteq \set{x \in \mathbb{R} \mid \text{for a real number } y,\, x = y^2}$.\\ Thus $\set{x \in \mathbb{R} \mid \text{for a real number } y,\, x = y^2} = \set{x \in \mathbb{R} \mid x \geq 0}$. $\qed$}
	\solution{$\set{x \in \mathbb{Z} \mid \text{for an integer }y,\, x = 6y} \subseteq \set{x \in \mathbb{Z} \mid \text{for an integer }y,\, x = 2y}$.}
	{Let $a \in \set{x \in \mathbb{Z} \mid \text{for an integer }y,\, x = 6y}$. Then $a = 6b$ for some integer $b$. Then $a = 2 \cdot 3b$. Since $b$ is an integer, $3b$ is also an integer. Therefore $a \in \set{x \in \mathbb{Z} \mid \text{for an integer }y,\, x = 2y}$. Thus $\set{x \in \mathbb{Z} \mid \text{for an integer }y,\, x = 6y} \subseteq \set{x \in \mathbb{Z} \mid \text{for an integer }y,\, x = 2y}$. $\qed$}
	\end{enumerate}
\item Prove each of the following for sets $A, B$, and $C$.
	\begin{enumerate}
	\solution{If $A \subseteq B$ and $B \subseteq C$, then $A \subseteq C$.}
	{Suppose $A \subseteq B$ and $B \subseteq C$. Then for any $a \in A$ we have that $a \in B$ and since $a \in B$ we have that $a \in C$. Thus $a \in C$ whenever $a \in A$ and therefore $A \subseteq C$.}
	\solution{If $A \subseteq B$ and $B \subset C$, then $A \subset C$.}
	{Suppose $A \subseteq B$ and $B \subset C$. Take the case when $A = B$. Since $B \subset C$ we have that $A \subset C$. Now take the case when $A \subset B$. For any $a \in A$ we have that $a \in B$. Since $B \subset C$ we have that $a \in C$. Then $A \subseteq C$. Since $B \subset C$, we know that there is some element $c \in C$ such that $c \notin B$ and since $A \subset B$ we know that $c \notin A$ and so $A \neq C$. Therefore $A \subset C$.}
	\solution{If $A \subset B$ and $B \subseteq C$, then $A \subset C$.}
	{Suppose $A \subset B$ and $B \subseteq C$. Take the case when $B = C$. Since $A \subset B$ we have that $A \subset C$. Now take the case when $B \subset C$. For any $a \in A$ we have that $a \in B$. Since $B \subset C$ we have that $a \in C$. Then $A \subseteq C$. Since $B \subset C$, we also know that there is some element $c \in C$ such that $c \notin B$ and since $A \subset B$ we know that $c \notin A$ and so $A \neq C$. Therefore $A \subset C$.}
	\solution{If $A \subset B$ and $B \subset C$, then $A \subset C$.}
	{Suppose $A \subset B$ and $B \subset C$. For any $a \in A$ we have that $a \in B$. Since $B \subset C$ we have that $a \in C$. Then $A \subset C$. Since $B \subset C$, we know that there is some element $c \in C$ such that $c \notin B$ and since $A \subset B$ we know that $c \notin A$ and so $A \neq C$. Therefore $A \subset C$.}
	\end{enumerate}
\solution{Give an example of sets $A,B,C,D$, and $E$ which satisfy the following conditions simultaneously: $A \subset B$, $B \in C$, $C \subset D$, and $D \subset E$.}
{Let $A = \emptyset$, $B = \set{\emptyset}$, $C = \set{\set{\emptyset}}$, $D = \set{\emptyset, \set{\emptyset}}$, and $E = \set{\emptyset, \set{\emptyset}, \set{\set{\emptyset}}}$.}
\item Which of the following are true for all sets $A, B$, and $C$?
	\begin{enumerate}
	\solution{If $A \notin B$ and $B \notin C$, then $A \notin C$.}
	{False: Let $A = \emptyset$, $B = \set{0}$ and $C = \set{\emptyset}$.}
	\solution{If $A \neq B$ and $B \neq C$, then $A \neq C$.}
	{False: Let $A = \mathbb{R}$, $B = \mathbb{Z}$ and $C = \mathbb{R}$.}
	\solution{If $A \in B$ and $B \not\subseteq C$, then $A \notin C$.}
	{False: Let $A = \emptyset$, $B = \set{\emptyset, 0}$ and $C = \set{\emptyset, 1}$.}
	\solution{If $A \subset B$ and $B \subseteq C$, then $C \not\subseteq A$.}
	{True: Suppose $A \subset B$ and $B \subseteq C$. Since $A \subset B$, there is some $b \in B$ such that $b \notin A$. Since $B \subseteq C$ we have that this $b \in C$ and thus there is a $c \in C$ such that $c \notin A$. Therefore $C \not\subseteq A$.}
	\solution{If $A \subset B$ and $B \subset C$, then $A \subset C$.}
	{True: Suppose $A \subset B$ and $B \subset C$. Then for any $a \in A$ we have that $a \in B$. Since $a \in B$ we have that $a \in C$. Therefore $A \subseteq C$. Since $B \subset C$ we have that there is some $c \in C$ such that $c \notin B$. Since $A \subset B$ we have that $c \notin A$. Therefore $A \neq C$ and so $A \subset C$.}
	\end{enumerate}
\solution{Show that for every set $A$, $A \subseteq \emptyset$ iff $A = \emptyset$.}
{($\Rightarrow$) Suppose $A \subseteq \emptyset$. Then for any $a \in A$ we have that $a \in \emptyset$. But since the empty set has no members, no such $a$ can exist. Therefore $A$ has no members and so $A = \emptyset$. ($\Leftarrow$) Suppose $A = \emptyset$. Then $A$ has no members and so, certainly, for all $a \in A$ we have that $a \in B$, for any set $B$. Therefore $A \subseteq B$. Letting $B = \emptyset$, we conclude that $A \subseteq \emptyset$. $\qed$}
\solution{Let $A_1,A_2,\dots,A_n$ be $n$ sets. Show that $$A_1 \subseteq A_2 \subseteq \dots \subseteq A_n \subseteq A_1 \text{ iff } A_1 = A_2 = \dots = A_n.$$}
{($\Leftarrow$) Suppose sets $A_1 = A_2 = \dots = A_n$. Then clearly $A_i \subseteq A_{i+1}$ for all $1 \leq i \leq n - 1$ and $A_n \subseteq A_1$. Therefore $A_1 \subseteq A_2 \subseteq \dots \subseteq A_n \subseteq A_1$.\\ ($\Rightarrow$) Suppose $A_1 \subseteq A_2 \subseteq \dots \subseteq A_n \subseteq A_1$. Then for any $a \in A_1$ we have that $a \in A_2$, $a \in A_3$, $\dots$, $a \in A_n$ and so $A_1 \subseteq A_n$. But since $A_n \subseteq A_1$ we must have that $A_1 = A_n$. Therefore $A_1 \subseteq A_2 \subseteq \dots \subseteq A_{n-1} \subseteq A_1$. Repeating this argument $n - 2$ times for $j = n - 1, n - 2, \dots, 2$ we find that for any $a \in A_1$ we have that $a \in A_j$ and therefore $A_1 \subseteq A_j$. But we also have that $A_j \subseteq A_1$ and so $A_1 = A_j$. Finally, we conclude that $A_1 = A_2 = \dots = A_n$. $\qed$}
\solution{Give several examples of a set $X$ such that each element of $X$ is a subset of $X$.}
{$X_1 = \emptyset$. $X_2 = \set{\emptyset}$. $X_3 = ?$}
\solution{List the members of $\mathcal{P}(A)$ if $A = \set{\set{1,2}, \set{3}, 1}$.}
{$\mathcal{P}(A) = \set{A, \set{\set{1,2}, \set{3}}, \set{\set{1,2}, 1}, \set{\set{3}, 1}, \set{\set{1,2}}, \set{\set{3}}, \set{1}, \emptyset}$}
\solution{For each positive integer $n$, give an example of a set $A_n$ of $n$ elements such that for each pair of elements of $A_n$, one member is an element of the other.}
{Let $A_0 = \emptyset$ and $A_1 = \set{A_0}$. For $n \geq 2$, let $A_n = \set{A_0, A_1, \dots, A_{n-1}}$.}
\end{enumerate}

\hrulefill

\renewcommand{\labelenumi}{4.\arabic{enumi}}
\begin{enumerate}
\solution{Prove that for all sets $A$ and $B$, $\emptyset \subseteq A \cap B \subseteq A \cup B$.}
{Let $A$ and $B$ be any sets. Since the empty set has no members, it's clear that for all $x \in \emptyset$ we have that $x \in C$ for any set $C$. Since $A \cap B$ is a set, we have that $\emptyset \subseteq A \cap B$. Now take any $x \in A \cap B$. Then $x \in A$ and $x \in B$ and so, clearly, $x \in A$ or $x \in B$. Therefore $x \in A \cup B$ and so $A \cap B \subseteq A \cup B$. Thus $\emptyset \subseteq A \cap B \subseteq A \cup B$.}
\solution{Let $\mathbb{Z}$ be the universal set, and let
\begin{align*}
A &= \set{x \in \mathbb{Z} \mid \text{for some positive integer } y,\, x = 2y},\\
B &= \set{x \in \mathbb{Z} \mid \text{for some positive integer } y,\, x = 2y - 1},\\
C &= \set{x \in \mathbb{Z} \mid x < 10}.
\end{align*}
Describe $\comp{A}, \comp{A \cup B}, A - \comp{C}$, and $C - (A \cup B)$, either in prose or by a defining property.}
{\begin{itemize}
\item $\comp{A} = \set{x \in \mathbb{Z} \mid x < 1} \cup B$
\item $\comp{A \cup B} = \set{x \in \mathbb{Z} \mid x < 1}$
\item $A - \comp{C} = \set{2,3,6,8}$
\item $C - (A \cup B) = \set{x \in \mathbb{Z} \mid x < 1}$
\end{itemize}}
\item Consider the following subsets of $\mathbb{Z^+}$, the set of positive integers:
\begin{align*}
A &= \set{x \in \mathbb{Z^+} \mid \text{for some integer } y,\, x = 2y},\\
B &= \set{x \in \mathbb{Z^+} \mid \text{for some integer } y,\, x = 2y + 1},\\
C &= \set{x \in \mathbb{Z^+} \mid \text{for some integer } y,\, x = 3y}.
\end{align*}
	\begin{enumerate}
	\solution{Describe $A \cap C, B \cup C$, and $B - C$.}
	{\begin{itemize}
		\item $A \cap C = \set{x \in \mathbb{Z^+} \mid \text{for some integer } y,\, x = 6y}$. This is the set of all positive multiples of both 2 and 3 (i.e. all positive multiples of 6).
		\item $B \cup C = B \cup \set{x \in \mathbb{Z^+} \mid \text{for some integer } y,\, x = 6y}$. This is the set of all positive odd integers along with all even positive multiples of 3 (i.e. all positive multiples of 6).
		\item $B - C = \set{x \in \mathbb{Z^+} \mid \text{for some integer } y,\, x = 3y + 1 \text{ or } x = 3y + 2}$. This is the set of all positive integers which are not divisible by 3.
	 \end{itemize}}
	\solution{Verify that $A \cap (B \cup C) = (A \cap B) \cup (A \cap C)$.}
	{In this example:
	\begin{itemize}
		\item $A \cap (B \cup C) = A \cap C$
		\item $A \cap B = \emptyset \Rightarrow (A \cap B) \cup (A \cap C) = A \cap C$
	\end{itemize}
	A general proof:\\
	Assume $x \in A \cap (B \cup C)$. Then by definition of set intersection, $x \in A$ and $x \in B \cup C$. By the definition of set union, $x \in B$ or $x \in C$. If $x \in B$ then we have that $x \in A \cap B$ since $x \in A$. Otherwise, if $x \in C$ then we have that $x \in A \cap C$ since $x \in A$. In both cases we know that $x \in (A \cap B) \cup (A \cap C)$ since both $A \cap B \subseteq (A \cap B) \cup (A \cap C)$ and $A \cap C \subseteq (A \cap B) \cup (A \cap C)$. Therefore $A \cap (B \cup C) \subseteq (A \cap B) \cup (A \cap C)$. Now assume $x \in (A \cap B) \cup (A \cap C)$. By the definition of set union, $x \in A \cap B$ or $x \in A \cap C$. If $x \in A \cap B$ then by the definition of set intersection, we have that $x \in A$ and $x \in B$. Since $B \subseteq B \cup C$ we know that $x \in B \cup C$. Thus $x \in A \cap (B \cup C)$. Otherwise if $x \in A \cap C$ then by the definition of set intersection, we have that $x \in A$ and $x \in C$. Since $C \subseteq B \cup C$ we know that $x \in B \cup C$. Thus $x \in A \cap (B \cup C)$. Therefore $(A \cap B) \cup (A \cap C) \subseteq  A \cap (B \cup C)$. Therefore $(A \cap B) \cup (A \cap C) =  A \cap (B \cup C)$. $\qed$}
	\end{enumerate}
\solution{If $A$ is any set, what are each of the following sets? $A \cap \emptyset$, $A \cup \emptyset$, $A - \emptyset$, $A - A$, $\emptyset - A$.}
{\begin{itemize}
\item $A \cap \emptyset = \set{x \mid x \in A \text{ and } x \in \emptyset} = \emptyset$.
\item $A \cup \emptyset = \set{x \mid x \in A \text{ or } x \in \emptyset} = A$.
\item $A - \emptyset = \set{x \mid x \in A \text{ and } x \notin \emptyset} = A$.
\item $A - A = \set{x \mid x \in A \text{ and } x \notin A} = \emptyset$.
\item $\emptyset - A = \set{x \mid x \in \emptyset \text{ and } x \notin A} = \emptyset$.
\end{itemize}}
\solution{Determine $\emptyset \cap \set{\emptyset}, \set{\emptyset} \cap \set{\emptyset}, \set{\emptyset, \set{\emptyset}} - \emptyset, \set{\emptyset, \set{\emptyset}} - \set{\emptyset}, \set{\emptyset,\set{\emptyset}} - \set{\set{\emptyset}}$}
{\begin{itemize}
\item $\emptyset \cap \set{\emptyset} = \set{x \mid x \in \emptyset \text{ and } x \in \set{\emptyset}} = \emptyset$.
\item $\set{\emptyset} \cap \set{\emptyset} = \set{x \mid x \in \set{\emptyset} \text{ and } x \in \set{\emptyset}} = \set{x \mid x \in \set{\emptyset}} = \set{\emptyset}$.
\item $\set{\emptyset, \set{\emptyset}} - \emptyset = \set{x \mid x \in \set{\emptyset, \set{\emptyset}} \text{ and } x \notin \emptyset} = \set{\emptyset, \set{\emptyset}}$.
\item $\set{\emptyset, \set{\emptyset}} - \set{\emptyset} = \set{x \mid x \in \set{\emptyset, \set{\emptyset}} \text{ and } x \notin \set{\emptyset}} = \set{\set{\emptyset}}$.
\item $\set{\emptyset,\set{\emptyset}} - \set{\set{\emptyset}} = \set{x \mid x \in \set{\emptyset,\set{\emptyset}} \text{ and } x \notin \set{\set{\emptyset}}} = \set{\emptyset}$.
\end{itemize}}
\solution{Suppose $A$ and $B$ are subsets of $U$. Show that in each of (a), (b), and (c) below, if any one of the relations stated holds, then each of the others holds.
\begin{enumerate}
\item $A \subseteq B, \comp{A} \supseteq \comp{B}, A \cup B = B, A \cap B = A$.
\item $A \cap B = \emptyset, A \subseteq \comp{B}, B \subseteq \comp{A}$.
\item $A \cup B = U, \comp{A} \subseteq B, \comp{B} \subseteq A$.}
\end{enumerate}
{\begin{enumerate}
\item \begin{enumerate}
		\item Assume $A \subseteq B$. 
		\begin{itemize}
			\item Take $x \notin B$. Assume $x \in A$. Since $A \subseteq B$ we have that $x \in B$. Then by contradiction we must have that $x \notin A$. Therefore $\comp{B} \subseteq \comp{A}$ or $\comp{A} \supseteq \comp{B}$.
			\item Take $x \in A \cup B$. Then $x \in A$ or $x \in B$. Assume $x \in A$. Then since $A \subseteq B$ we have that $x \in B$. Therefore $A \cup B \subseteq B$. But since $B \subseteq A \cup B$ we have that $A \cup B = B$.
			\item Take $x \in A$. Because $A \subseteq B$ we have that $x \in B$. Since $x \in A$ and $x \in B$ we have that $x \in A \cap B$ and therefore $A \subseteq A \cap B$. But since $A \cap B \subseteq A$, we have $A \cap B = A$.
		\end{itemize}
		\item Assume $\comp{A} \supseteq \comp{B}$.
		\begin{itemize}
		\item Take $x \in A$. Assume $x \notin B$. Then because $\comp{A} \supseteq \comp{B}$ we have that $x \notin A$. Then by contradiction we must have that $x \in B$. Therefore $A \subseteq B$.
		\item Take $x \in A \cup B$. Then by the definition of set union, $x \in A$ or $x \in B$. Assume $x \notin B$. Then because $\comp{A} \supseteq \comp{B}$ we have that $x \notin A$. But then since $x \notin A$ and $x \notin B$ we have that $x \notin A \cup B$. By contradiction we must have that $x \in B$ and thus $A \cup B \subseteq B$. But since $B \subseteq A \cup B$ we have $A \cup B = B$.
		\item Take $x \in A$. Assume $x \notin B$. Then because $\comp{A} \supseteq \comp{B}$ we have that $x \notin A$. By contradiction, $x \in B$. Thus $A \subseteq A \cap B$. But since $A \cap B \subseteq A$ we have $A \cap B = A$.
		\end{itemize}
		\item Assume $A \cup B = B$.
		\begin{itemize}
		\item Take $x \in A$. Then $x \in A \cup B = B$ and therefore $A \subseteq B$.
		\item Take $x \notin B$. Then $x \notin A \cup B$ and so $x \notin A$. Therefore $\comp{A} \supseteq \comp{B}$.
		\item Take $x \in A$. Then $x \in A \cup B = B$. Since $x \in A$ and $x \in B$ then $x \in A \cap B$ and so $A \subseteq A \cap B$. But since $A \cap B \subseteq A$ we have $A \cap B = A$.
		\end{itemize}
		\item Assume $A \cap B = A$.
		\begin{itemize}
		\item Take $x \in A$. Then $x \in A \cap B$ and so $x \in B$. Therefore $A \subseteq B$.
		\item Take $x \notin B$. Then $x \notin A \cap B$ and so $x \notin A$. Therefore $\comp{A} \supseteq \comp{B}$.
		\item Take $x \in A \cup B$. Then $x \in A$ or $x \in B$. Assume $x \notin B$. Then $x \notin A \cap B = A$ and so $x \notin A \cup B$. By contradiction, we must have the $x \in B$. Then $A \cup B \subseteq B$. But since $B \subseteq A \cup B$ we have that $A \cup B = B$.
		\end{itemize}
		\end{enumerate}
\item \begin{enumerate}
		\item Assume $A \cap B = \emptyset$.
		\begin{itemize}
			\item Take $x \in A$. Since $A \cap B = \emptyset$, $x \notin B$. Then $A \subseteq \comp{B}$.
			\item Take $x \in B$. Since $A \cap B = \emptyset$, $x \notin A$. Then $B \subseteq \comp{A}$.
		\end{itemize}
		\item Assume $A \subseteq \comp{B}$.
		\begin{itemize}
			\item Take $x \in A$. Then since $A \subseteq \comp{B}$ we have $x \notin B$. Now take $x \in B$. Assume $x \in A$. But then since $A \subseteq \comp{B}$ we have $x \notin B$. By contradiction we must have that $x \notin A$. Thus $x \in A \Rightarrow x \notin B$ and $x \in B \Rightarrow x \notin A$ and so $A \cap B = \emptyset$.
			\item Take $x \in B$. Assume $x \in A$. Then since $A \subseteq \comp{B}$ we have that $x \notin B$. Then by contradiction, we must have $x \notin A$ and so $B \subseteq \comp{A}$.
		\end{itemize}
		\item Assume $B \subseteq \comp{A}$.
		\begin{itemize}
			\item Take $x \in B$. Then since $B \subseteq \comp{A}$ we have $x \notin A$. Now take $x \in A$ and assume $x \in B$. But since $B \subseteq \comp{A}$ we have $x \notin A$. Then by contradiction, we must have that $x \notin B$. Thus $x \in A \Rightarrow x \notin B$ and $x \in B \Rightarrow x \notin A$ and so $A \cap B = \emptyset$.
			\item Take $x \in A$. Assume $x \in B$. Then since $B \subseteq \comp{A}$ we have $x \notin A$. Then by contradiction we must have that $x \notin B$. Therefore $A \subseteq \comp{B}$.
		\end{itemize}
	  \end{enumerate}
\item \begin{enumerate}
		\item Assume $A \cup B = U$.
		\begin{itemize}
			\item Take $x \notin A$. Assume $x \notin B$. Then since $x \notin A$ and $x \notin B$, we have that $x \notin A \cup B = U$. But since $A \subseteq U$ this is a contradiction. So we must have that $x \in B$. Therefore $\comp{A} \subseteq B$.
			\item Take $x \notin B$. Assume $x \notin A$. Then since $x \notin A$ and $x \notin B$, we have that $x \notin A \cup B = U$. But since $B \subseteq U$ this is a contradiction. So we must have that $x \in A$. Therefore $\comp{B} \subseteq A$.
		\end{itemize}
		\item Assume $\comp{A} \subseteq B$.
		\begin{itemize}
			\item Take $x \in U$. Then either $x \in A$ or $x \notin A$. Assume that $x \in A$. Then $x \in A \cup B$. Now assume that $x \notin A$. Then since $\comp{A} \subseteq B$ we have that $x \in B$. Then $x \in A \cup B$. Therefore $U \subseteq A \cup B$. But since $A \cup B \subseteq U$ we must have that $A \cup B = U$. 
			\item Take $x \notin B$. Assume$ \comp{B} \subseteq A$
		\end{itemize}
		\item Assume $\comp{B} \subseteq A$.
		\begin{itemize}
			\item Take $x \in U$. Then $x \in B$ or $x \notin B$. If $x \in B$ then $x \in A \cup B$. If $x \notin B$ then since $\comp{B} \subseteq A$ we have that $x \in A$ and so $x \in A \cup B$. Therefore $U \subseteq A \cup B$. But since $A \cup B \subseteq U$ we must have that $A \cup B = U$.
			\item Take $x \notin A$ and assume $x \notin B$. Then since $\comp{B} \subseteq A$ we have that $x \in A$. Then by contradiction we must have that $x \in B$ and so $\comp{A} \subseteq B$.
		\end{itemize}
	\end{enumerate}
\end{enumerate}}
\solution{Prove that for all sets $A,B$, and $C$, $$(A \cap B) \cup C = A \cap (B \cup C) \text{ iff } C \subseteq A.$$}
{($\Rightarrow$) Assume $(A \cap B) \cup C = A \cap (B \cup C)$. Let $x \in C$. Then $x \in (A \cap B) \cup C$. Since $(A \cap B) \cup C = A \cap (B \cup C)$ we have that $x \in A \cap (B \cup C)$ and thus $x \in A$. Therefore $C \subseteq A$.\\ ($\Leftarrow$) Assume $C \subseteq A$. Let $x \in (A \cap B) \cup C$. Then $x \in A \cap B$ or $x \in C$. Assume $x \in A \cap B$. Then $x \in A$ and $x \in B$. Since $x \in B$ then $x \in B \cup C$. Since $x \in A$ and $x \in B \cup C$ we have that $x \in A \cap (B \cup C)$. Now assume $x \in C$. Then $x \in B \cup C$. Since $C \subseteq A$ we also have that $x \in A$. Thus $x \in A \cap (B \cup C)$. Therefore $(A \cap B) \cup C \subseteq A \cap (B \cup C)$. Now let $x \in A \cap (B \cup C)$. Then $x \in A$ and $x \in B \cup C$. Then $x \in B$ or $x \in C$. Assume $x \in B$. Then since $x \in A$ and $x \in B$ we have that $x \in A \cap B$. Therefore $x \in (A \cap B) \cup C$. Now assume $x \in C$. Then $x \in (A \cap B) \cup C$. Therefore $A \cap (B \cup C) \subseteq (A \cap B) \cup C$ and so we can conclude that $(A \cap B) \cup C = A \cap (B \cup C)$. $\qed$}
\solution{Prove that for all sets $A,B$, and $C$, $$(A - B) - C = (A - C) - (B - C).$$}
{Let $x \in (A - B) - C$. Then $x \in A - B$ and $x \notin C$. Since $x \in A - B$ we have that $x \in A$ and $x \notin B$. Then since $x \in A$ and $x \notin C$ we have that $x \in A - C$. Since $x \notin B$ we have that $x \notin B - C$. Then since $x \in A - C$ and $x \notin B - C$ we have that $x \in (A - C) - (B - C)$. Now let $x \in (A - C) - (B - C)$. Then $x \in A - C$ and $x \notin B - C$. Then since $x \in A - C$ we have that $x \in A$ and $x \notin C$. Since $x \notin B - C$ we have that either $x \notin B$ or $x \in C$. But $x \in C$ is a contradiction because $x \notin C$. Therefore $x \notin B$. Then since $x \in A$ and $x \notin B$ we have that $x \in A - B$ and since $x \notin C$ we have that $x \in (A - B) - C$. Therefore $(A - C) - (B - C) \subseteq (A - B) - C$ and thus $(A - B) - C = (A - C) - (B - C)$. $\qed$}
\item{\begin{enumerate}
\solution{Draw the Venn diagram of the symmetric difference, $A + B$, of sets $A$ and $B$.}
{... TO DO ...}
\solution{Using a Venn diagram, show that the symmetric difference is a commutative and associative operation.}
{... TO DO ...}
\solution{Show that for every set $A$, $A + A = \emptyset$ and $A + \emptyset = A$}
{Let $A$ be a set. Let $x \in A + A = (A - A) \cup (A - A)$. Then $x \in A - A = \set{x \mid x \in A \text{ and } x \notin A} = \emptyset$. Therefore $A + A \subseteq \emptyset$. Then since $\emptyset \subseteq A + A$ we have that $A + A = \emptyset$. $\qed$\\ Let $x \in A + \emptyset = (A - \emptyset) \cup (\emptyset - A)$. Then $x \in A - \emptyset$ or $x \in \emptyset - A$. Assume $x \in \emptyset - A$. Then $x \in \emptyset$. But this is a contradiction and so we have that $x \in A - \emptyset$. Then $x \in A$. Therefore $A + \emptyset \subseteq A$. Now let $x \in A$. Since $x \notin \emptyset$ we have that $x \in A - \emptyset$. Then $x \in (A - \emptyset) \cup (\emptyset - A) = A + \emptyset$. Thus $A \subseteq A + \emptyset$ and therefore $A + \emptyset = A$. $\qed$}
\end{enumerate}}
\solution{The Venn diagram for subsets $A, B$, and $C$ of $U$, in general, divides the rectangle representing $U$ into eight nonoverlapping regions. Label each region with a combination of $A, B$, and $C$ which represents exactly that region.}
{... TO DO ...}
\item{With the aid of a Venn diagram investigate the validity of each of the following inferences:
\begin{enumerate}
\solution{If $A, B$, and $C$ are subsets of $U$ such that $A \cap B \subseteq \comp{C}$ and $A \cup C \subseteq B$, then $A \cap C = \emptyset$.}
{... TO DO ...}
\solution{If $A, B$, and $C$ are subsets of $U$ such that $A \subseteq \comp{B \cup C}$ and $B \subseteq \comp{A \cup C}$, then $B = \emptyset$.}
{... TO DO ...}
\end{enumerate}}
\end{enumerate}

\hrulefill

\renewcommand{\labelenumi}{5.\arabic{enumi}}
\begin{enumerate}
\solution{Prove that parts 3', 4', and 5' of Theorem 5.1 are identities}
{\renewcommand{\labelenumii}{\arabic{enumii}'.}
\begin{enumerate}
\setcounter{enumii}{2}
\item Assume $x \in A \cap (B \cup C)$. Then $x \in A$ and $x \in B \cup C$. If $x \in B$ then since $x \in A$ we have $x \in A \cap B$ and so $x \in (A \cap B) \cup (A \cap C)$. Otherwise if $x \in C$ then since $x \in A$ we have $x \in A \cap C$ and so $x \in (A \cap B) \cup (A \cap C)$. Therefore $A \cap (B \cup C) \subseteq (A \cap B) \cup (A \cap C)$. Now assume $x \in (A \cap B) \cup (A \cap C)$. Then $x \in A \cap B$ or $x \in A \cap C$. If $x \in A \cap B$ then $x \in A$ and $x \in B$. Since $x \in B$ we have $x \in B \cup C$. Since we also have $x \in A$ then $x \in A \cap (B \cup C)$. Otherwise if $x \in A \cap C$ then $x \in A$ and $x \in C$. Since $x \in C$ we have $x \in B \cup C$. Since we also have $x \in A$ then $x \in A \cap (B \cup C)$. Therefore $(A \cap B) \cup (A \cap C) \subseteq A \cap (B \cup C)$. Hence $A \cap (B \cup C) = (A \cap B) \cup (A \cap C)$.\\
\item Assume $x \in A \cap U$. Then $x \in A$ and $x \in U$. Therefore $A \cap U \subseteq A$. Now assume $x \in A$. Then since $A \subseteq U$ we have $x \in U$ and so $x \in A \cap U$. Therefore $A \subseteq A \cap U$. Hence $A \cap U = A$.\\
\item Assume $x \in A \cap \comp{A}$. Then $x \in A$ and $x \in \comp{A}$. Since $x \in \comp{A}$ we have $x \notin A$. Since $x \in A$ and $x \notin A$ we have $x \in \emptyset$. Therefore $A \cap \comp{A} \subseteq \emptyset$. Since $\emptyset \subseteq X$ for any set $X$ we have $\emptyset \subseteq A \cap \comp{A}$. Hence $A \cap \comp{A} = \emptyset$.
\end{enumerate}}
\solution{Prove the unprimed parts of Theorem 5.2 using the membership relations. Try to prove the same results using only Theorem 5.1. In at least one such proof write out the dual of each step to demonstrate that a proof of the dual results.}
{\renewcommand{\labelenumii}{\arabic{enumii}.}
\begin{enumerate}
\setcounter{enumii}{5}
\item %6%
\underline{using membership}: Assume that $A \cup B = A$ for all $A$. Then, in particular, for $A = \emptyset$ we have $A \cup B = \emptyset \cup B = \emptyset$. Take $x \in B$. Then $x \in \emptyset \cup B = \emptyset$ and so $B \subseteq \emptyset$. Now take $x \in \emptyset$. By ex falso quodlibet, we have $x \in B$. Therefore $\emptyset \subseteq B$. Thus $B = \emptyset$.\\ \underline{using Thm 5.1}: Assume that $A \cup B = A$ for all $A$. Then, in particular, $\emptyset \cup B = \emptyset$. Then \begin{align*}B &= B \cup \emptyset\tag{5.1.4}\\&= \emptyset \cup B\tag{5.1.2}\\&= \emptyset.\tag{Consequence of assumption}\end{align*} Therefore $B = \emptyset$.\\ \underline{dual proof}: Assume that $A \cap B = A$ for all $A$. Then, in particular, $U \cap B = U$. Then \begin{align*}B &= B \cap U\tag{5.1.4'}\\&= U \cap B\tag{5.1.2'}\\&= U.\tag{Consequence of assumption}\end{align*} Therefore $B = U$.
\item %7%
\underline{using membership}: Assume $A \cup B = U$ and $A \cap B = \emptyset$. Take $x \in B$. Assume $x \in A$. Then $x \in A \cap B = \emptyset$. By contradiction, $x \notin A$ and so $x \in \set{y \in U \mid y \notin A} = U - A = \comp{A}$. Therefore $B \subseteq \comp{A}$. Now take $x \in \comp{A}$. Then $x \notin A$. Assume $x \notin B$. Then $x \notin A \cup B = U$. By contradiction, $x \in B$ and so $\comp{A} \subseteq B$. Therefore $B = \comp{A}$.\\ 
\underline{using Thm 5.1}: Assume $A \cup B = U$ and $A \cap B = \emptyset$. Then we have \begin{align*}B &= B \cap U \tag{5.1.4'}\\&= B \cap (A \cup \comp{A})\tag{5.1.5}\\&= (B \cap A) \cup (B \cap \comp{A})\tag{5.1.3'}\\&= (A \cap B) \cup (B \cap \comp{A})\tag{5.1.2'}\\&=(A \cap B) \cup (\comp{A} \cap B)\tag{5.1.2'}\\&= \emptyset \cup (\comp{A} \cap B) \tag{Assumption}\\&= (A \cap \comp{A}) \cup (\comp{A} \cap B)\tag{5.1.5'}\\&= (\comp{A} \cap A) \cup (\comp{A} \cap B)\tag{5.1.2'}\\&= \comp{A} \cap (A \cup B)\tag{5.1.3'}\\&= \comp{A} \cap U\tag{Assumption}\\&= \comp{A}.\tag{5.1.4'}\end{align*}(self-dual)
\item %8%
\underline{using membership}: Take $x \in \comp{\comp{A}}$. Then $x \in U - \comp{A} = \set{y \in U \mid y \notin \comp{A}} = \set{y \in U \mid y \in A} = A$. Therefore $\comp{\comp{A}} \subseteq A$. Now take $x \in A$. Then $x \notin \comp{A}$ and so $x \in \comp{\comp{A}}$. Then $A \subseteq \comp{\comp{A}}$ and so $\comp{\comp{A}} = A$.\\
\underline{using Thm 5.1}: \begin{align*}\comp{\comp{A}} &= \comp{\comp{A}} \cap U \tag{5.1.4'}\\&= \comp{\comp{A}} \cap (A \cup \comp{A}) \tag{5.1.5}\\&= (\comp{\comp{A}} \cap A) \cup (\comp{\comp{A}} \cap \comp{A}) \tag{5.1.3'}\\&= (\comp{\comp{A}} \cap A) \cup (\comp{A} \cap \comp{\comp{A}}) \tag{5.1.2'}\\&= (\comp{\comp{A}} \cap A) \cup \emptyset \tag{5.1.5'}\\&= \emptyset \cup (\comp{\comp{A}} \cap A)\tag{5.1.2}\\&= (A \cap \comp{A}) \cup (\comp{\comp{A}} \cap A)\tag{5.1.5'}\\&= (A \cap \comp{A}) \cup (A \cap \comp{\comp{A}})\tag{5.1.2'}\\&= A \cap (\comp{A} \cup \comp{\comp{A}})\tag{5.1.3'}\\&= A \cap U\tag{5.1.5}\\&= A.\tag{5.1.4'}\end{align*} (self-dual)
\item %9%
\underline{using membership}: Take $x \in \comp{\emptyset}$. Then $x \in U - \emptyset = \set{y \in U \mid y \notin \emptyset} = \set{y \in U} = U$. Therefore $\comp{\emptyset} \subseteq U$. Now take $x \in U$. Then $x \notin \emptyset$ and so $x \in \comp{\emptyset}$. Therefore $U \subseteq \comp{\emptyset}$. Thus $\comp{\emptyset} = U$.\\
\underline{using Thm 5.1}: \begin{align*}\comp{\emptyset} &= \comp{\emptyset} \cup \emptyset \tag{5.1.4}\\&= \emptyset \cup \comp{\emptyset} \tag{5.1.2}\\&= U.\tag{5.1.5}\end{align*}
\item %10%
\underline{using membership}: Take $x \in A \cup A$. Then $x \in A$ or $x \in A$. Obviously, $x \in A$ and so $A \cup A \subseteq A$. Take $x \in A$. Then $x \in A \cup A$ and so $A \subseteq A \cup A$. Thus $A \cup A = A$.\\
\underline{using Thm 5.1}: \begin{align*}A \cup A &= (A \cup A) \cap U \tag{5.1.4'}\\&= (A \cap A) \cap (A \cup \comp{A})\tag{5.1.5}\\&= A \cup (A \cap \comp{A})\tag{5.1.3}\\&= A \cup \emptyset\tag{5.1.5'}\\&= A.\tag{5.1.4}\end{align*}
\item %11%
\underline{using membership}: Take $x \in A \cup U$. Then $x \in A$ or $x \in U$. Assume $x \notin U$. Then $x \notin A$ since $A \subseteq U$. By contradiction, we must have $x \in U$. Therefore $A \cup U \subseteq U$. Now take $x \in U$. Then $x \in U \cup A = A \cup U$ and therefore $U \subseteq A \cup U$. Thus $A \cup U = U$.\\
\underline{using Thm 5.1}: \begin{align*}A \cup U &= (A \cup U) \cap U\tag{5.1.4'}\\&= U \cap (A \cup U)\tag{5.1.2'}\\&= (A \cup \comp{A}) \cap (A \cup U)\tag{5.1.5}\\&= A \cup (\comp{A} \cap U)\tag{5.1.3}\\&= A \cup \comp{A}\tag{5.1.4'}\\&= U.\tag{5.1.5}\end{align*}
\item %12%
\underline{using membership}: Take $x \in A \cup (A \cap B)$. Then $x \in A$ or $x \in A \cap B$. In the first case, $x \in A$. In the second case, $x \in A$ and $x \in B$. In both cases $x \in A$ and so $A \cup (A \cap B) \subseteq A$. Now take $x \in A$. Then $x \in A \cup (A \cap B)$ and so $A \subseteq A \cup (A \cap B)$. Thus $A \cup (A \cap B) = A$.\\
\underline{using Thm 5.1}: $A = A \cup \emptyset = A \cup ((A \cap B) \cap (\comp{A \cap B})) = A \cup ((A \cap B) \cap (\comp{A} \cup \comp{B})) = (A \cup (A \cap B)) \cap (A \cup (\comp{A} \cup \comp{B})) = (A \cup (A \cap B)) \cap ((A \cup \comp{A}) \cup \comp{B}) = (A \cup (A \cap B)) \cap (U \cup \comp{B}) = (A \cup (A \cap B)) \cap (\comp{B} \cup U) = (A \cup (A \cap B)) \cap U = A \cup (A \cap B)$.
\item %13%
\underline{using membership}: Take $x \in \comp{A \cup B}$. Then $x \notin A \cup B$. Then $x \notin A$ and $x \notin B$ and so $x \in \comp{A}$ and $x \in \comp{B}$ and therefore $x \in \comp{A} \cap \comp{B}$. Thus $\comp{A \cup B} \subseteq \comp{A} \cap \comp{B}$. Now take $x \in \comp{A} \cap \comp{B}$. Then $x \in \comp{A}$ and $x \in \comp{B}$ and so $x \notin A$ and $x \notin B$. Therefore $x \notin A \cup B$ and so $x \in \comp{A \cup B}$. Then we have $\comp{A} \cap \comp{B} \subseteq \comp{A \cup B}$. Thus $\comp{A \cup B} = \comp{A} \cap \comp{B}$.\\
\underline{using Thm 5.1}: \begin{align*}A \cup (A \cap B) &= (A \cap U) \cup (A \cap B)\tag{5.1.4'}\\&= (A \cap (B \cup \comp{B})) \cup (A \cap B)\tag{5.1.5}\\&= ((A \cap B) \cup (A \cap \comp{B})) \cup (A \cap B)\tag{5.1.3'}\\&= (A \cap B) \cup ((A \cap \comp{B}) \cup (A \cap B))\tag{5.1.1}\\&= (A \cap B) \cup ((A \cap B) \cup (A \cap \comp{B}))\tag{5.1.2}\\&= ((A \cap B) \cup (A \cap B)) \cup (A \cap \comp{B})\tag{5.1.1}\\&= (((A \cap B) \cup (A \cap B)) \cap U) \cup (A \cap \comp{B})\tag{5.1.4'}\\&=  (((A \cap B) \cup (A \cap B)) \cap ((A \cap B) \cup (\comp{A \cap B}))) \cup (A \cap \comp{B})\tag{5.1.5}\\&= ((A \cap B) \cup ((A \cap B) \cap (\comp{A \cap B}))) \cup (A \cap \comp{B})\tag{5.1.3}\\&= ((A \cap B) \cup \emptyset) \cup (A \cap \comp{B})\tag{5.1.5'}\\&= (A \cap B) \cup (A \cap \comp{B})\tag{5.1.4}\\&= A \cap (B \cup \comp{B})\tag{5.1.3'}\\&= A \cap U\tag{5.1.5}\\&= A.\tag{5.1.4'}\end{align*}
\end{enumerate}}

\renewcommand{\labelenumii}{(\alph{enumii})}
\item Using only the identities in Theorems 5.1 and 5.2, show that each of the following equations is an identity.
\begin{enumerate}
	\solution{$(A \cap B \cap X) \cup (A \cap B \cap C \cap X \cap Y) \cup (A \cap X \cap \comp{A}) = A \cap B \cap X$.}
	{\begin{align*}&(A \cap B \cap X) \cup (A \cap B \cap C \cap X \cap Y) \cup (A \cap X \cap \comp{A})\\&= [(A \cap B \cap X) \cup (A \cap B \cap C \cap X \cap Y)] \cup (A \cap X \cap \comp{A})\\&= [(A \cap B \cap X) \cup (A \cap B \cap X \cap C \cap Y)] \cup (A \cap X \cap \comp{A})\\&= [(A \cap B \cap X) \cup ((A \cap B \cap X) \cap (C \cap Y))] \cup (A \cap X \cap \comp{A})\\&= (A \cap B \cap X) \cup (A \cap X \cap \comp{A})\\&= (A \cap B \cap X) \cup (A \cap \comp{A} \cap X)\\&= (A \cap B \cap X) \cup (\emptyset \cap X)\\&= (A \cap B \cap X) \cup (X \cap \emptyset)\\&= (A \cap B \cap X) \cup \emptyset\\&= A \cap B \cap X.\end{align*}}
	\solution{$(A \cap B \cap C) \cup (\comp{A} \cap B \cap C) \cup \comp{B} \cup \comp{C} = U$.}
	{\begin{align*}&(A \cap B \cap C) \cup (\comp{A} \cap B \cap C) \cup \comp{B} \cup \comp{C}\\&= (B \cap C \cap A) \cup (B \cap C \cap \comp{A}) \cup \comp{B} \cup \comp{C}\\&= [(B \cap C) \cap A] \cup [(B \cap C) \cap \comp{A}] \cup \comp{B} \cup \comp{C}\\&= [(B \cap C) \cap (A \cup \comp{A})] \cup \comp{B} \cup \comp{C}\\&= [(B \cap C) \cap U] \cup \comp{B} \cup \comp{C}\\&= (B \cap C) \cup \comp{B} \cup \comp{C}\\&= (B \cap C) \cup (\comp{B \cap C})\\&= U.\end{align*}}
	\solution{$(A \cap B \cap C \cap \comp{X}) \cup (\comp{A} \cap C) \cup (\comp{B} \cap C) \cup (C \cap X) = C.$}
	{\begin{align*}&(A \cap B \cap C \cap \comp{X}) \cup (\comp{A} \cap C) \cup (\comp{B} \cap C) \cup (C \cap X)\\&= (C \cap A \cap B \cap \comp{X}) \cup (C \cap \comp{A}) \cup (C \cap \comp{B}) \cup (C \cap X)\\&= (C \cap A \cap B \cap \comp{X}) \cup (C \cap (\comp{A} \cup \comp{B})) \cup (C \cap X)\\&= (C \cap A \cap B \cap \comp{X}) \cup (C \cap [(\comp{A} \cup \comp{B}) \cup X])\\&= (C \cap A \cap B \cap \comp{X}) \cup (C \cap (\comp{A} \cup \comp{B} \cup X))\\&= (C \cap A \cap B \cap \comp{X}) \cup (C \cap (\comp{A \cap B \cup \comp{X}}))\\&= (C \cap (A \cap B \cap \comp{X})) \cup (C \cap (\comp{A \cap B \cup \comp{X}}))\\&= C \cap [(A \cap B \cap \comp{X}) \cup (\comp{A \cap B \cup \comp{X}})]\\&= C \cap U\\&= C.\end{align*}}\pagebreak
	\solution{$[(A \cap B) \cup (A \cap C) \cup (\comp{A} \cap \comp{X} \cap Y)] \cap [\comp{(A \cap \comp{B} \cap C) \cup (\comp{A} \cap \comp{X} \cap \comp{Y}) \cup (\comp{A} \cap B \cap Y)}] = (A \cap B) \cup (\comp{A} \cap \comp{B} \cap \comp{X} \cap Y)$.}
	{In the words of George Costanza: ``Let's get nuts!''
\begin{align*}&[(A \cap B) \cup (A \cap C) \cup (\comp{A} \cap \comp{X} \cap Y)] \cap [\comp{(A \cap \comp{B} \cap C) \cup (\comp{A} \cap \comp{X} \cap \comp{Y}) \cup (\comp{A} \cap B \cap Y)}] \\&= [(A \cap B) \cup (A \cap C) \cup (\comp{A} \cap \comp{X} \cap Y)] \cap [\comp{(A \cap \comp{B} \cap C) \cup (\comp{A} \cap \comp{X} \cap \comp{Y})} \cap (\comp{\comp{A} \cap B \cap Y})]\tag{5.2.13}\\&= [(A \cap B) \cup (A \cap C) \cup (\comp{A} \cap \comp{X} \cap Y)] \cap [(\comp{A \cap \comp{B} \cap C}) \cap (\comp{\comp{A} \cap \comp{X} \cap \comp{Y}}) \cap (\comp{\comp{A} \cap B \cap Y})] \tag{5.2.13}\\&= [(A \cap B) \cup (A \cap C) \cup (\comp{A} \cap \comp{X} \cap Y)] \cap [(\comp{A} \cup \comp{\comp{B}} \cup \comp{C}) \cap (\comp{\comp{A}} \cup \comp{\comp{X}} \cup \comp{\comp{Y}}) \cap (\comp{\comp{A}} \cup \comp{B} \cup \comp{Y})]\tag{5.2.13'}\\&= [(A \cap B) \cup (A \cap C) \cup (\comp{A} \cap \comp{X} \cap Y)] \cap [(\comp{A} \cup B \cup \comp{C}) \cap (A \cup X \cup Y) \cap (A \cup \comp{B} \cup \comp{Y})]\tag{5.2.8}\\&= [(A \cap B) \cup (A \cap C) \cup (\comp{A} \cap \comp{X} \cap Y)] \cap [(\comp{A} \cup B \cup \comp{C}) \cap (A \cup \comp{B} \cup \comp{Y}) \cap (A \cup X \cup Y)]\tag{5.1.2'}\\&= [(A \cap B) \cup (A \cap C) \cup (\comp{A} \cap \comp{X} \cap Y)] \cap [(A \cup \comp{B} \cup \comp{Y}) \cap (\comp{A} \cup B \cup \comp{C}) \cap (A \cup X \cup Y)]\tag{5.1.2'}\\&= [(A \cap B) \cup (A \cap C) \cup (\comp{A} \cap \comp{X} \cap Y)] \cap [(A \cup \comp{B} \cup \comp{Y})  \cap (A \cup X \cup Y) \cap (\comp{A} \cup B \cup \comp{C})]\tag{5.1.2'}\\&= [[[(A \cap B) \cup (A \cap C) \cup (\comp{A} \cap \comp{X} \cap Y)] \cap (A \cup \comp{B} \cup \comp{Y})] \cap (A \cup X \cup Y)] \cap (\comp{A} \cup B \cup \comp{C})\tag{5.1.1'}\end{align*}
I will now break the derivation into three parts, taking care of each of the nested intersections, and taking some reasonable shortcuts along the way. In most places, I've gone above and beyond to make sure that every step appeals to theorem 5.1 or 5.2.\pagebreak
\begin{align*}&[(A \cap B) \cup (A \cap C) \cup (\comp{A} \cap \comp{X} \cap Y)] \cap (A \cup \comp{B} \cup \comp{Y})\\&= (A \cup \comp{B} \cup \comp{Y}) \cap [(A \cap B) \cup (A \cap C) \cup (\comp{A} \cap \comp{X} \cap Y)]\\&= [(A \cup \comp{B} \cup \comp{Y}) \cap (A \cap B)] \cup [(A \cup \comp{B} \cup \comp{Y}) \cap (A \cap C)] \cup [(A \cup \comp{B} \cup \comp{Y}) \cap (\comp{A} \cap \comp{X} \cap Y)]\\&= [((A \cup \comp{B} \cup \comp{Y}) \cap A) \cap B] \cup [((A \cup \comp{B} \cup \comp{Y}) \cap A) \cap C)] \cup [(A \cup \comp{B} \cup \comp{Y}) \cap (\comp{A} \cap \comp{X} \cap Y)]\\&= [(A \cap (A \cup \comp{B} \cup \comp{Y})) \cap B] \cup [(A \cap (A \cup \comp{B} \cup \comp{Y})) \cap C)] \cup [(A \cup \comp{B} \cup \comp{Y}) \cap (\comp{A} \cap \comp{X} \cap Y)]\\&= [(A \cap (A \cup (\comp{B} \cup \comp{Y}))) \cap B] \cup [(A \cap (A \cup (\comp{B} \cup \comp{Y}))) \cap C)] \cup [(A \cup \comp{B} \cup \comp{Y}) \cap (\comp{A} \cap \comp{X} \cap Y)]\\&= (A \cap B) \cup (A \cap C) \cup [(A \cup \comp{B} \cup \comp{Y}) \cap (\comp{A} \cap \comp{X} \cap Y)]\\&= (A \cap B) \cup (A \cap C) \cup [((A \cup \comp{B} \cup \comp{Y}) \cap \comp{A}) \cap \comp{X} \cap Y)]\\&= (A \cap B) \cup (A \cap C) \cup [(\comp{A} \cap (A \cup \comp{B} \cup \comp{Y})) \cap \comp{X} \cap Y)]\\&= (A \cap B) \cup (A \cap C) \cup [((\comp{A} \cap A) \cup (\comp{A} \cap \comp{B}) \cup (\comp{A} \cap \comp{Y})) \cap \comp{X} \cap Y)]\\&= (A \cap B) \cup (A \cap C) \cup [((A \cap \comp{A}) \cup (\comp{A} \cap \comp{B}) \cup (\comp{A} \cap \comp{Y})) \cap \comp{X} \cap Y)]\\&= (A \cap B) \cup (A \cap C) \cup [(\emptyset \cup (\comp{A} \cap \comp{B}) \cup (\comp{A} \cap \comp{Y})) \cap \comp{X} \cap Y)]\\&= (A \cap B) \cup (A \cap C) \cup [((\comp{A} \cap \comp{B}) \cup (\comp{A} \cap \comp{Y}) \cup \emptyset) \cap \comp{X} \cap Y)]\\&= (A \cap B) \cup (A \cap C) \cup [((\comp{A} \cap \comp{B}) \cup (\comp{A} \cap \comp{Y})) \cap \comp{X} \cap Y)]\\&= (A \cap B) \cup (A \cap C) \cup [((\comp{A} \cap \comp{B}) \cup \comp{A}) \cap ((\comp{A} \cap \comp{B}) \cup \comp{Y})) \cap \comp{X} \cap Y)]\\&= (A \cap B) \cup (A \cap C) \cup [(\comp{A} \cup (\comp{A} \cap \comp{B})) \cap (\comp{Y} \cup (\comp{A} \cap \comp{B})) \cap \comp{X} \cap Y)]\\&= (A \cap B) \cup (A \cap C) \cup [(\comp{A} \cap (\comp{Y} \cup (\comp{A} \cap \comp{B})) \cap \comp{X} \cap Y)]\\&= (A \cap B) \cup (A \cap C) \cup [(\comp{A} \cap (\comp{Y} \cup \comp{A}) \cap (\comp{Y} \cup \comp{B})) \cap \comp{X} \cap Y)]\\&= (A \cap B) \cup (A \cap C) \cup [(\comp{A} \cap (\comp{A} \cup \comp{Y}) \cap (\comp{Y} \cup \comp{B})) \cap \comp{X} \cap Y)]\\&= (A \cap B) \cup (A \cap C) \cup [(\comp{A} \cap (\comp{Y} \cup \comp{B})) \cap \comp{X} \cap Y)]\\&= (A \cap B) \cup (A \cap C) \cup [\comp{A} \cap ((\comp{Y} \cup \comp{B}) \cap \comp{X}) \cap Y)]\\&= (A \cap B) \cup (A \cap C) \cup [\comp{A} \cap (\comp{X} \cap (\comp{Y} \cup \comp{B})) \cap Y)]\\&= (A \cap B) \cup (A \cap C) \cup [\comp{A} \cap \comp{X} \cap ((\comp{Y} \cup \comp{B}) \cap Y)]\\&= (A \cap B) \cup (A \cap C) \cup [\comp{A} \cap \comp{X} \cap (Y \cap (\comp{Y} \cup \comp{B}))]\\&= (A \cap B) \cup (A \cap C) \cup [\comp{A} \cap \comp{X} \cap ((Y \cap \comp{Y}) \cup (Y \cap \comp{B}))]\\&= (A \cap B) \cup (A \cap C) \cup [\comp{A} \cap \comp{X} \cap (\emptyset \cup (Y \cap \comp{B}))]\\&= (A \cap B) \cup (A \cap C) \cup [\comp{A} \cap \comp{X} \cap ((Y \cap \comp{B})\cup \emptyset)]\\&= (A \cap B) \cup (A \cap C) \cup [\comp{A} \cap \comp{X} \cap (Y \cap \comp{B})]\\&= (A \cap B) \cup (A \cap C) \cup [\comp{A} \cap \comp{X} \cap (\comp{B} \cap Y)]\\&= (A \cap B) \cup (A \cap C) \cup [\comp{A} \cap (\comp{X} \cap \comp{B}) \cap Y]\\&= (A \cap B) \cup (A \cap C) \cup [\comp{A} \cap (\comp{B} \cap \comp{X}) \cap Y]\\&= (A \cap B) \cup (A \cap C) \cup (\comp{A} \cap \comp{B} \cap \comp{X} \cap Y).\end{align*}
\begin{align*}&[(A \cap B) \cup (A \cap C) \cup (\comp{A} \cap \comp{B} \cap \comp{X} \cap Y)] \cap (A \cup X \cup Y)\\&= (A \cup X \cup Y) \cap [(A \cap B) \cup (A \cap C) \cup (\comp{A} \cap \comp{B} \cap \comp{X} \cap Y)]\\&=[(A \cup X \cup Y) \cap (A \cap B)] \cup [(A \cup X \cup Y) \cap (A \cap C)] \cup [(A \cup X \cup Y) \cap (\comp{A} \cap \comp{B} \cap \comp{X} \cap Y)]\\&= [((A \cup X \cup Y) \cap A) \cap B] \cup [((A \cup X \cup Y) \cap A)\cap C)] \cup [(A \cup X \cup Y) \cap (\comp{A} \cap \comp{B} \cap \comp{X} \cap Y)]\\&= [(A \cap (A \cup X \cup Y)) \cap B] \cup [(A \cap (A \cup X \cup Y))\cap C)] \cup [(A \cup X \cup Y) \cap (\comp{A} \cap \comp{B} \cap \comp{X} \cap Y)]\\&= (A \cap B) \cup (A \cap C) \cup [(A \cup X \cup Y) \cap (\comp{A} \cap \comp{B} \cap \comp{X} \cap Y)]\\&= (A \cap B) \cup (A \cap C) \cup [(A \cup X \cup Y) \cap (Y \cap \comp{A} \cap \comp{B} \cap \comp{X})]\\&= (A \cap B) \cup (A \cap C) \cup [((A \cup X \cup Y) \cap Y) \cap \comp{A} \cap \comp{B} \cap \comp{X}]\\&= (A \cap B) \cup (A \cap C) \cup [(Y \cap (A \cup X \cup Y)) \cap \comp{A} \cap \comp{B} \cap \comp{X}]\\&= (A \cap B) \cup (A \cap C) \cup [(Y \cap (Y \cup A \cup X)) \cap \comp{A} \cap \comp{B} \cap \comp{X}]\\&= (A \cap B) \cup (A \cap C) \cup (Y \cap \comp{A} \cap \comp{B} \cap \comp{X})\\&= (A \cap B) \cup (A \cap C) \cup (\comp{A} \cap \comp{B} \cap \comp{X} \cap Y).\end{align*}
\begin{align*}&[(A \cap B) \cup (A \cap C) \cup (\comp{A} \cap \comp{B} \cap \comp{X} \cap Y)] \cap (\comp{A} \cup B \cup \comp{C})\\&= [(A \cap B) \cap (\comp{A} \cup B \cup \comp{C})] \cup [(A \cap C) \cap (\comp{A} \cup B \cup \comp{C})] \cup [(\comp{A} \cap \comp{B} \cap \comp{X} \cap Y) \cap (\comp{A} \cup B \cup \comp{C})]\\&= [A \cap (B \cap (\comp{A} \cup B \cup \comp{C}))] \cup [(A \cap C) \cap (\comp{A} \cup B \cup \comp{C})] \cup [(\comp{A} \cap \comp{B} \cap \comp{X} \cap Y) \cap (\comp{A} \cup B \cup \comp{C})]\\&= [A \cap (B \cap (B \cup \comp{A} \cup \comp{C}))] \cup [(A \cap C) \cap (\comp{A} \cup B \cup \comp{C})] \cup [(\comp{A} \cap \comp{B} \cap \comp{X} \cap Y) \cap (\comp{A} \cup B \cup \comp{C})]\\&= (A \cap B) \cup [(A \cap C) \cap (\comp{A} \cup B \cup \comp{C})] \cup [(\comp{A} \cap \comp{B} \cap \comp{X} \cap Y) \cap (\comp{A} \cup B \cup \comp{C})]\\&= (A \cap B) \cup [(A \cap C) \cap (\comp{A} \cup \comp{C} \cup B)] \cup [(\comp{A} \cap \comp{B} \cap \comp{X} \cap Y) \cap (\comp{A} \cup B \cup \comp{C})]\\&= (A \cap B) \cup [(A \cap C) \cap (\comp{A \cap C} \cup B)] \cup [(\comp{A} \cap \comp{B} \cap \comp{X} \cap Y) \cap (\comp{A} \cup B \cup \comp{C})]\\&= (A \cap B) \cup [((A \cap C) \cap (\comp{A \cap C})) \cup ((A \cap C) \cap B)] \cup [(\comp{A} \cap \comp{B} \cap \comp{X} \cap Y) \cap (\comp{A} \cup B \cup \comp{C})]\\&= (A \cap B) \cup [\emptyset \cup ((A \cap C) \cap B)] \cup [(\comp{A} \cap \comp{B} \cap \comp{X} \cap Y) \cap (\comp{A} \cup B \cup \comp{C})]\\&= (A \cap B) \cup [((A \cap C) \cap B) \cup \emptyset] \cup [(\comp{A} \cap \comp{B} \cap \comp{X} \cap Y) \cap (\comp{A} \cup B \cup \comp{C})]\\&= (A \cap B) \cup ((A \cap C) \cap B) \cup [(\comp{A} \cap \comp{B} \cap \comp{X} \cap Y) \cap (\comp{A} \cup B \cup \comp{C})]\\&= (A \cap B) \cup (A \cap (C \cap B)) \cup [(\comp{A} \cap \comp{B} \cap \comp{X} \cap Y) \cap (\comp{A} \cup B \cup \comp{C})]\\&= (A \cap B) \cup (A \cap (B \cap C)) \cup [(\comp{A} \cap \comp{B} \cap \comp{X} \cap Y) \cap (\comp{A} \cup B \cup \comp{C})]\\&= (A \cap B) \cup ((A \cap B) \cap C) \cup [(\comp{A} \cap \comp{B} \cap \comp{X} \cap Y) \cap (\comp{A} \cup B \cup \comp{C})]\\&= (A \cap B) \cup [(\comp{A} \cap \comp{B} \cap \comp{X} \cap Y) \cap (\comp{A} \cup B \cup \comp{C})]\\&= (A \cap B) \cup [((\comp{A} \cap \comp{B} \cap \comp{X} \cap Y) \cap \comp{A}) \cup ((\comp{A} \cap \comp{B} \cap \comp{X} \cap Y) \cap B) \cup ((\comp{A} \cap \comp{B} \cap \comp{X} \cap Y) \cap \comp{C})]\\&= (A \cap B) \cup [(\comp{A} \cap (\comp{A} \cap \comp{B} \cap \comp{X} \cap Y)) \cup (B \cap (\comp{A} \cap \comp{B} \cap \comp{X} \cap Y)) \cup (\comp{C} \cap (\comp{A} \cap \comp{B} \cap \comp{X} \cap Y))]\\&= (A \cap B) \cup [((\comp{A} \cap \comp{A}) \cap \comp{B} \cap \comp{X} \cap Y) \cup (B \cap (\comp{A} \cap \comp{B} \cap \comp{X} \cap Y)) \cup (\comp{C} \cap (\comp{A} \cap \comp{B} \cap \comp{X} \cap Y))]\\&= (A \cap B) \cup [(\comp{A} \cap \comp{B} \cap \comp{X} \cap Y) \cup (B \cap (\comp{A} \cap \comp{B} \cap \comp{X} \cap Y)) \cup (\comp{C} \cap (\comp{A} \cap \comp{B} \cap \comp{X} \cap Y))]\\&= (A \cap B) \cup [(\comp{A} \cap \comp{B} \cap \comp{X} \cap Y) \cup (B \cap (\comp{B} \cap \comp{A} \cap \comp{X} \cap Y)) \cup (\comp{C} \cap (\comp{A} \cap \comp{B} \cap \comp{X} \cap Y))]\\&= (A \cap B) \cup [(\comp{A} \cap \comp{B} \cap \comp{X} \cap Y) \cup ((B \cap \comp{B}) \cap \comp{A} \cap \comp{X} \cap Y) \cup (\comp{C} \cap (\comp{A} \cap \comp{B} \cap \comp{X} \cap Y))]\\&= (A \cap B) \cup [(\comp{A} \cap \comp{B} \cap \comp{X} \cap Y) \cup (\emptyset \cap \comp{A} \cap \comp{X} \cap Y) \cup (\comp{C} \cap (\comp{A} \cap \comp{B} \cap \comp{X} \cap Y))]\\&= (A \cap B) \cup [(\comp{A} \cap \comp{B} \cap \comp{X} \cap Y) \cup (\comp{A} \cap \comp{X} \cap Y \cap \emptyset) \cup (\comp{C} \cap (\comp{A} \cap \comp{B} \cap \comp{X} \cap Y))]\\&= (A \cap B) \cup [(\comp{A} \cap \comp{B} \cap \comp{X} \cap Y) \cup \emptyset \cup (\comp{C} \cap (\comp{A} \cap \comp{B} \cap \comp{X} \cap Y))]\\&= (A \cap B) \cup [(\comp{A} \cap \comp{B} \cap \comp{X} \cap Y) \cup (\comp{C} \cap (\comp{A} \cap \comp{B} \cap \comp{X} \cap Y)) \cup \emptyset]\\&= (A \cap B) \cup [(\comp{A} \cap \comp{B} \cap \comp{X} \cap Y) \cup (\comp{C} \cap (\comp{A} \cap \comp{B} \cap \comp{X} \cap Y))]\\&= (A \cap B) \cup [(\comp{A} \cap \comp{B} \cap \comp{X} \cap Y) \cup ((\comp{A} \cap \comp{B} \cap \comp{X} \cap Y) \cap \comp{C})]\\&= (A \cap B) \cup (\comp{A} \cap \comp{B} \cap \comp{X} \cap Y).\end{align*}}
\end{enumerate}

\solution{Rework Exercise 4.9(b), using solely the algebra of sets developed in this section.}
{Exercise 4.9(b) states: Using a Venn diagram, show that the symmetric difference is a commutative and associative operation. In the language of the algebra of sets, this translates to: For all sets $A,B$ and $C$ show that $A + B = B + A$ and that $A + (B + C) = (A + B) + C$.
\begin{itemize}
\item \thm{Theorem}: $A + B = B + A$.\\
\thm{Proof}: \begin{align*}A + B &= (A - B) \cup (B - A)\tag{Definition of symmetric difference}\\&=
(B - A) \cup (A - B)\tag{5.1.2}\\&= 
B + A.\tag{Definition of symmetric difference}\end{align*}
\item \thm{Theorem}: $A + (B + C) = (A + B) + C$.\\
In order to prove the theorem, the following few lemmas will be useful:\\
\thm{Lemma 1}: $(A - B) - C = A - (B \cup C)$.\\
\thm{Proof}: \begin{align*}(A - B) - C &= (A \cap \comp{B}) - C \tag{Definition of relative complement}\\&= (A \cap \comp{B}) \cap \comp{C}\tag{Definition of relative complement}\\&= A \cap (\comp{B} \cap \comp{C}) \tag{5.1.1'}\\&= A \cap (\comp{B \cup C})\tag{5.1.13}\\&= A - (B \cup C)\tag{Definition of relative complement}\end{align*} 

\thm{Lemma 2}: $(A - C) \cup (B - C) = (A \cup B) - C$.\\
\thm{Proof}: \begin{align*}(A - C) \cup (B - C) &= (A \cap \comp{C}) \cup (B \cap \comp{C})\tag{Definition of relative complement}\\&= (\comp{C} \cap A) \cup (\comp{C} \cap B)\tag{5.1.2'}\\&= \comp{C} \cap (A \cup B)\tag{5.1.3'}\\&= (A \cup B) \cap \comp{C}\tag{5.1.2'}\\&= (A \cup B) - C \tag{Definition of relative complement}\end{align*}

\thm{Lemma 3}: $(A \cup \comp{B}) \cap (\comp{A} \cup B) = (A \cap B) \cup (\comp{A} \cap \comp{B})$.\\
\thm{Proof}: \begin{align*} (A \cup \comp{B}) \cap (\comp{A} \cup B) &= [(A \cup \comp{B}) \cap \comp{A}] \cup [(A \cup \comp{B}) \cap B] \tag{5.1.3'}\\&= [\comp{A} \cap (A \cup \comp{B})] \cup [B \cap (A \cup \comp{B})]\tag{5.1.2'}\\&= [(\comp{A} \cap A) \cup (\comp{A} \cap \comp{B})] \cup [(B \cap A) \cup (B \cap \comp{B})]\tag{5.1.3'}\\&= [(A \cap \comp{A}) \cup (\comp{A} \cap \comp{B})] \cup [(A \cap B) \cup (B \cap \comp{B})]\tag{5.1.2'}\\&= [(\comp{A} \cap \comp{B}) \cup (A \cap \comp{A})] \cup [(A \cap B) \cup (B \cap \comp{B})]\tag{5.1.2}\\&= [(A \cap B) \cup (B \cap \comp{B})] \cup [(\comp{A} \cap \comp{B}) \cup (A \cap \comp{A})]\tag{5.1.2}\\&= [(A \cap B) \cup \emptyset] \cup [(\comp{A} \cap \comp{B}) \cup \emptyset]\tag{5.1.5'}\\&= (A \cap B) \cup (\comp{A} \cap \comp{B})\tag{5.1.4}\end{align*}

\thm{Proof of theorem}:
\begin{align*} A + (B + C) &= A + [(B - C) \cup (C - B)]\tag{Definition of symmetric difference}\\&= 
(A - [(B - C) \cup (C - B)]) \cup ([(B - C) \cup (C - B)] - A)\tag{Definition of symmetric difference}\\&= 
([A - (B - C)] - [C - B]) \cup ([(B - C) \cup (C - B)] - A)\tag{Lemma 1}\\&= 
([A - (B - C)] - [C - B]) \cup ([(B - C) - A] \cup [(C - B) - A])\tag{Lemma 2}\\&= 
([A - (B \cap \comp{C})] - [C \cap \comp{B}]) \cup ([(B \cap \comp{C}) - A] \cup [(C \cap \comp{B}) - A])\tag{Definition of relative complement}\\&= 
([A - (B \cap \comp{C})] - [C \cap \comp{B}]) \cup ([(B \cap \comp{C}) \cap \comp{A}] \cup [(C \cap \comp{B}) \cap \comp{A}])\tag{Definition of relative complement}\\&= 
([A - (B \cap \comp{C})] - [C \cap \comp{B}]) \cup ([\comp{A} \cap (B \cap \comp{C})] \cup [\comp{A} \cap (C \cap \comp{B})])\tag{5.1.2'}\\&= 
([A - (B \cap \comp{C})] - [C \cap \comp{B}]) \cup ([\comp{A} \cap (B \cap \comp{C})] \cup [\comp{A} \cap (\comp{B} \cap C)])\tag{5.1.2'}\\&= 
([A - (B \cap \comp{C})] - [C \cap \comp{B}]) \cup (\comp{A} \cap B \cap \comp{C}) \cup (\comp{A} \cap \comp{B} \cap C)\tag{5.1.1'}\\&= 
(A - [(B \cap \comp{C}) \cup (C \cap \comp{B})]) \cup (\comp{A} \cap B \cap \comp{C}) \cup (\comp{A} \cap \comp{B} \cap C)\tag{Lemma 1}\\&= 
(A \cap [\comp{(B \cap \comp{C}) \cup (C \cap \comp{B})}]) \cup (\comp{A} \cap B \cap \comp{C}) \cup (\comp{A} \cap \comp{B} \cap C)\tag{Definition of relative complement}\\&= 
(A \cap [(\comp{B \cap \comp{C}}) \cap (\comp{C \cap \comp{B}})]) \cup (\comp{A} \cap B \cap \comp{C}) \cup (\comp{A} \cap \comp{B} \cap C)\tag{5.2.13}\\&= 
(A \cap [(\comp{B} \cup \comp{\comp{C}}) \cap (\comp{C} \cup \comp{\comp{B}})]) \cup (\comp{A} \cap B \cap \comp{C}) \cup (\comp{A} \cap \comp{B} \cap C)\tag{5.2.13'}\\&= 
(A \cap [(\comp{B} \cup C) \cap (\comp{C} \cup B)]) \cup (\comp{A} \cap B \cap \comp{C}) \cup (\comp{A} \cap \comp{B} \cap C)\tag{5.2.8}\\&= 
(A \cap [(\comp{B} \cup C) \cap (B \cup \comp{C})]) \cup (\comp{A} \cap B \cap \comp{C}) \cup (\comp{A} \cap \comp{B} \cap C)\tag{5.1.2}\\&= 
(A \cap [(B \cap C) \cup (\comp{B} \cap \comp{C})]) \cup (\comp{A} \cap B \cap \comp{C}) \cup (\comp{A} \cap \comp{B} \cap C)\tag{Lemma 3}\\&= 
[(A \cap (B \cap C)) \cup (A \cap (\comp{B} \cap \comp{C}))] \cup (\comp{A} \cap B \cap \comp{C}) \cup (\comp{A} \cap \comp{B} \cap C)\tag{5.1.3'}
\end{align*} 
 
\begin{align*}&=[(A \cap B \cap C) \cup (A \cap \comp{B} \cap \comp{C})] \cup (\comp{A} \cap B \cap \comp{C}) \cup (\comp{A} \cap \comp{B} \cap C)\tag{5.1.1'}\\&= 
(A \cap B \cap C) \cup (A \cap \comp{B} \cap \comp{C}) \cup (\comp{A} \cap B \cap \comp{C}) \cup (\comp{A} \cap \comp{B} \cap C)\tag{5.1.1}\\&= 
(A \cap \comp{B} \cap \comp{C}) \cup (A \cap B \cap C) \cup (\comp{A} \cap B \cap \comp{C}) \cup (\comp{A} \cap \comp{B} \cap C)\tag{5.1.2}\\&=
(A \cap \comp{B} \cap \comp{C}) \cup (\comp{A} \cap B \cap \comp{C}) \cup (A \cap B \cap C) \cup (\comp{A} \cap \comp{B} \cap C)\\&= 
[(A \cap \comp{B} \cap \comp{C}) \cup (\comp{A} \cap B \cap \comp{C})] \cup [(A \cap B \cap C) \cup (\comp{A} \cap \comp{B} \cap C)]\tag{5.1.1}\\&= 
[(A \cap \comp{B} \cap \comp{C}) \cup (\comp{A} \cap B \cap \comp{C})] \cup [(C \cap A \cap B) \cup (C \cap \comp{A} \cap \comp{B})]\tag{5.1.2'}\\&= 
[(A \cap \comp{B} \cap \comp{C}) \cup (\comp{A} \cap B \cap \comp{C})] \cup [C \cap [(A \cap B) \cup (\comp{A} \cap \comp{B})]]\tag{5.1.3'}\\&= 
[(A \cap \comp{B} \cap \comp{C}) \cup (\comp{A} \cap B \cap \comp{C})] \cup [C \cap [(A \cup \comp{B}) \cap (\comp{A} \cup B)]]\tag{Lemma 3}\\&= 
[(A \cap \comp{B} \cap \comp{C}) \cup (\comp{A} \cap B \cap \comp{C})] \cup [[C \cap (A \cup \comp{B})] \cap (\comp{A} \cup B)]\tag{5.1.1'}\\&= 
[(A \cap \comp{B} \cap \comp{C}) \cup (\comp{A} \cap B \cap \comp{C})] \cup [[C \cap (\comp{A} \cup B)] \cap (A \cup \comp{B})]\tag{5.1.2'}\\&=
[(A \cap \comp{B} \cap \comp{C}) \cup (\comp{A} \cap B \cap \comp{C})] \cup [[C \cap (\comp{A} \cup B)] \cap (\comp{\comp{A}} \cup \comp{B})]\tag{5.2.8}\\&=
[(A \cap \comp{B} \cap \comp{C}) \cup (\comp{A} \cap B \cap \comp{C})] \cup [[C \cap (\comp{A} \cup B)] \cap (\comp{\comp{A} \cap B})]\tag{5.2.13'}\\&=
[(A \cap \comp{B} \cap \comp{C}) \cup (\comp{A} \cap B \cap \comp{C})] \cup [[C \cap (\comp{A} \cup B)] - (\comp{A} \cap B)]\tag{Definition of relative complement}\\&=
[(A \cap \comp{B} \cap \comp{C}) \cup (\comp{A} \cap B \cap \comp{C})] \cup [[C \cap (\comp{A} \cup \comp{\comp{B}})] - (\comp{A} \cap B)]\tag{5.2.8}\\&=
[(A \cap \comp{B} \cap \comp{C}) \cup (\comp{A} \cap B \cap \comp{C})] \cup [[C \cap (\comp{A \cap \comp{B}})] - (\comp{A} \cap B)]\tag{5.2.13'}\\&=
[(A \cap \comp{B} \cap \comp{C}) \cup (\comp{A} \cap B \cap \comp{C})] \cup [[C - (A \cap \comp{B})] - (\comp{A} \cap B)]\tag{Definition of relative complement}\\&=
[(A \cap \comp{B} \cap \comp{C}) \cup (\comp{A} \cap B \cap \comp{C})] \cup [[C - (A \cap \comp{B})] - (B \cap \comp{A})]\tag{5.1.2'}\\&=
[(A \cap \comp{B} \cap \comp{C}) \cup (\comp{A} \cap B \cap \comp{C})] \cup [[C - (A - B)] - (B - A)]\tag{Definition of relative complement}\\&=
[(A \cap \comp{B} \cap \comp{C}) \cup (\comp{A} \cap B \cap \comp{C})] \cup [C - [(A - B) \cup (B - A)]]\tag{Lemma 1}\\&=
[((A \cap \comp{B}) \cap \comp{C}) \cup ((\comp{A} \cap B) \cap \comp{C})] \cup [C - [(A - B) \cup (B - A)]]\tag{5.1.1'}\\&=
[((A \cap \comp{B}) - C) \cup ((\comp{A} \cap B) - C)] \cup [C - [(A - B) \cup (B - A)]]\tag{Definition of relative complement}\\&=
[((A \cap \comp{B}) - C) \cup ((B \cap \comp{A}) - C)] \cup [C - [(A - B) \cup (B - A)]]\tag{5.1.2'}\\&=
[((A - B) - C) \cup ((B - A) - C)] \cup [C - [(A - B) \cup (B - A)]]\tag{Definition of relative complement}\\&=
[[(A - B) \cup (B - A)] - C] \cup [C - [(A - B) \cup (B - A)]]\tag{Lemma 2}\\&=
[(A + B) - C] \cup [C - (A + B)]\tag{Definition of symmetric difference}\\&=
(A + B) + C\tag{Definition of symmetric difference}
\end{align*}
\end{itemize}}
\pagebreak
\solution{Let $A_1, A_2, \dots, A_n$ be sets, and define $S_k$ to be $A_1 \cup A_2 \cup \dots \cup A_k$ for $k = 1, 2, \dots, n$. Show that $$\mathcal{A} = \set{A_1, A_2 - S_1, A_3 - S_2, \dots, A_n - S_n}$$ is a disjoint collection of sets and that $$S_n = \set{A_1 \cup (A_2 - S_1) \cup \dots \cup (A_n - S_{n-1})}.$$ When is $\mathcal{A}$ a partition of $S_n$?}
{First we will extend the definition of $S_k$ to include the case of\linebreak $k=0$: Define $S_0 = \emptyset$. Then we may write $A_1 = A_1 - S_0$.\\ Now take any two distinct elements $B_1 = A_i - S_{i-1}$ and $B_2 = A_j - S_{j-1}$ of $\mathcal{A}$ where, without loss of generality, $1 \leq i < j \leq n$. Then $S_{j-1} = A_1 \cup \dots \cup A_i \cup \dots \cup A_{j-1}$ and so we have $$B_2 = A_j - S_{j-1} = A_j \cap (\comp{A_1 \cup \dots \cup A_i \cup \dots \cup A_{j-1}}) = A_j \cap \comp{A_1} \cap \dots \cap \comp{A_i} \cap \dots \cap \comp{A_{j-1}}.$$ Therefore $B_1 \cap B_2 = (A_i - S_{i-1}) \cap (A_j - S_{j-1}) = (A_i \cap \comp{S_{i-1}}) \cap (A_j \cap \comp{A_1} \cap \dots \cap \comp{A_i} \cap \dots \cap \comp{A_{j-1}}) = A_i \cap \comp{A_i} \cap (\comp{S_{i-1}} \cap \comp{A_1} \cap \dots \cap \comp{A_{i-1}} \cap \comp{A_{i+1}} \cap \dots \cap \comp{A_{j-1}}) = \emptyset \cap (\comp{S_{i-1}} \cap \comp{A_1} \cap \dots \cap \comp{A_{i-1}} \cap \comp{A_{i+1}} \cap \dots \cap \comp{A_{j-1}}) = \emptyset$ and thus $\mathcal{A}$ is a disjoint collection of sets.\\

To prove $S_n = \set{A_1 \cup (A_2 - S_1) \cup \dots \cup (A_n - S_{n-1})}$, first we show that \begin{align*}A_1 \cup (A_2 - S_1) &= A_1 \cup (A_2 \cap \comp{S_1})\\&= A_1 \cup (A_2 \cap \comp{A_1})\\&= (A_1 \cup A_2) \cap (A_1 \cap \comp{A_1})\\&= (A_1 \cup A_2) \cap \emptyset\\&= A_1 \cup A_2.\end{align*} Now suppose $A_1 \cup (A_2 - S_1) \cup \dots \cup (A_k - S_{k-1}) = A_1 \cup A_2 \cup \dots \cup A_k$ for some $k>2$. Then \begin{align*}A_1 \cup (A_2 - S_1) &\cup \dots \cup (A_k - S_{k-1}) \cup (A_{k+1} - S_k)\\&= A_1 \cup A_2 \cup \dots \cup A_k \cup (A_{k+1} - S_k)\\&= A_1 \cup A_2 \cup \dots \cup A_k \cup (A_{k+1} \cap \comp{S_k})\\&= A_1 \cup A_2 \cup \dots \cup A_k \cup (A_{k+1} \cap \comp{A_1 \cup A_2 \cup \dots \cup A_k})\\&= (A_1 \cup A_2 \cup \dots \cup A_k \cup A_{k+1}) \cap (A_1 \cup A_2 \cup \dots \cup A_k \cup \comp{A_1 \cup A_2 \cup \dots \cup A_k})\\&= (A_1 \cup A_2 \cup \dots \cup A_k \cup A_{k+1}) \cap U\\&= A_1 \cup A_2 \cup \dots \cup A_k \cup A_{k+1}.\end{align*} Therefore for any given $n$, $S_n = \set{A_1 \cup (A_2 - S_1) \cup \dots \cup (A_n - S_{n-1})}$.\\

$\mathcal{A}$ is a partition of $S_n$ as long as all of $A_i$ for $1\leq i \leq n$ are nonempty. (This is all we need since we already have that $\mathcal{A}$ is a disjoint collection of sets.) We have that $A_i$ is empty whenever $A_i \subseteq A_j$ for some $j < i$ since $A_i - S_{i-1} = A_i \cap (\comp{A_1} \cap \dots \cap \comp{A_j} \cap \dots \cap \comp{A_{i-1}}) = A_i \cap \comp{A_j} \cap \dots = \emptyset \cap \dots = \emptyset.$}

\solution{Prove that for arbitrary sets $A_1, A_2, \dots, A_n$ ($n \geq 2$), \begin{align*}A_1 \cup A_2 \cup \dots \cup A_n = (A_1 - A_2) \cup & (A_2 - A_3) \cup \dots \cup (A_{n-1} - A_n)\\&\cup (A_n - A_1) \cup (A_1 \cap A_2 \cap \dots \cap A_n).\end{align*}}
{Take $x \in A_1 \cup A_2 \cup \dots \cup A_n$. Then $x \in A_i$ for some $1 \leq i \leq n$. If $x \notin A_{i+1}$ then we have $x \in (A_i - A_{i+1})$ (if $i=n$ substitute 1 for $i+1$ - i.e.\ arithmetic on the set indices is modulo $n$). Otherwise, if $x \notin A_{i+2}$, then $x \in (A_{i+1} - A_{i+2})$. If not, continue in this manner, checking whether $x \in A_{i+3},A_{i+4}$, etc. If for all $1 \leq j \leq n$ we have $x \in A_j$ but $x \notin (A_j - A_{j+1})$ then we must have $x \in A_1 \cap A_2 \cap \dots \cap A_n$. In any case, $x \in (A_1 - A_2) \cup (A_2 - A_3) \cup \dots \cup (A_{n-1} - A_n) \cup (A_n - A_1) \cup (A_1 \cap A_2 \cap \dots \cap A_n)$ and so $A_1 \cup A_2 \cup \dots \cup A_n \subseteq (A_1 - A_2) \cup (A_2 - A_3) \cup \dots \cup (A_{n-1} - A_n) \cup (A_n - A_1) \cup (A_1 \cap A_2 \cap \dots \cap A_n)$. Now take $x \in (A_1 - A_2) \cup (A_2 - A_3) \cup \dots \cup (A_{n-1} - A_n) \cup (A_n - A_1) \cup (A_1 \cap A_2 \cap \dots \cap A_n)$. Then $x \in A_i - A_{i+1}$ for some $1 \leq i \leq n$ (again substituting 1 for $i+1$ when $i=n$) or $x \in A_1 \cap A_2 \cap \dots \cap A_n$. If $x \in A_1 \cap A_2 \cap \dots \cap A_n$ then clearly $x \in A_1 \cup A_2 \cup \dots \cup A_n$. Now if $x \in A_i - A_{i+1}$ for some $i$ then clearly $x \in A_i$ and so $x \in A_1 \cup A_2 \cup \dots \cup A_n$. Therefore $(A_1 - A_2) \cup (A_2 - A_3) \cup \dots \cup (A_{n-1} - A_n) \cup (A_n - A_1) \cup (A_1 \cap A_2 \cap \dots \cap A_n) \subseteq A_1 \cup A_2 \cup \dots \cup A_n$ and thus $A_1 \cup A_2 \cup \dots \cup A_n = (A_1 - A_2) \cup (A_2 - A_3) \cup \dots \cup (A_{n-1} - A_n) \cup (A_n - A_1) \cup (A_1 \cap A_2 \cap \dots \cap A_n)$.}

\item Referring to Example 5.2, prove the following.
\begin{enumerate}
\solution{For all sets $A$ and $B$, $A = B$ iff $A + B = \emptyset$.}
{Let $A = B$. Then $A + B = (A - B) \cup (B - A) = (A - A) \cup (B - B) = \emptyset \cup \emptyset = \emptyset$. Now let $A + B = \emptyset$. Then $(A - B) \cup (B - A) = \emptyset$. Thus $A - B = \emptyset$ and so $A = B$.}

\solution{An equation in $X$ with righthand member $\emptyset$ can be reduced to one of the form $(A \cap X) \cup (B \cap \comp{X}) = \emptyset$. (Suggestion: Sketch a proof along these lines. First, apply the DeMorgan laws until only complements of individual sets appear. Then expand the resulting lefthand side by the distributive law 3 so as to transform it into the union of several terms $T_n$ each of which is an intersection of several individual sets. Next, if in any $T_i$, neither $X$ nor $\comp{X}$ appears, replace $T_i$ by $T_i \cap (X \cup \comp{X})$ and expand. Finally, group together the terms containing $X$ and those containing $\comp{X}$ and apply distributive law 3'.)}
{...}

\solution{For all sets $A$ and $B$, $A = B = \emptyset$ iff $A \cup B = \emptyset$.}
{Let $A = B = \emptyset$. Then $A \cup B = \emptyset \cup \emptyset = \emptyset$. Now let $A \cup B = \emptyset$. Then $A = \emptyset$ and $B = \emptyset$.}

\solution{The equation $(A \cap X) \cup (B \cap \comp{X}) = \emptyset$ has a solution iff $B \subseteq \comp{A}$, and then any $X$ such that $B \subseteq X \subseteq \comp{A}$ is a solution.}
{($\Rightarrow$) Suppose there exists a set $X$ for which $(A \cap X) \cup (B \cap \comp{X}) = \emptyset$. Now consider some $x \in B$ and assume $x \in A$. Then if $x \in X$ we have $x \in A \cap X \subseteq (A \cap X) \cup (B \cap \comp{X}) = \emptyset$, a contradiction. Otherwise, $x \in \comp{X}$ and so $x \in B \cap \comp{X} \subseteq (A \cap X) \cup (B \cap \comp{X}) = \emptyset$, another contradiction. Therefore, $x \notin A$ and so $B \subseteq \comp{A}$.\\ ($\Leftarrow$) Now assuming $B \subseteq \comp{A}$ consider a set $X$ such that $B \subseteq X \subseteq \comp{A}$. Then for any $x \in X$, $x \in \comp{A} \Rightarrow x \notin A \Rightarrow X \cap A = \emptyset$. Also, for any $x \in B$, $x \in X$. Therefore $x \notin \comp{X}$ and so $B \cap \comp{X} = \emptyset$. Therefore $(A \cap X) \cup (B \cap \comp{X}) = \emptyset$.}

\solution{An alternative form for solutions of the equation in part (d) is $X = (B \cup T) \cap \comp{A}$, where $T$ is an arbitrary set.}
{Let $X = (B \cup T) \cap \comp{A}$ for an arbitrary set $T$. Then we have $\comp{X} = (\comp{B} \cap \comp{T}) \cup A$, $A \cap X = A \cap [(B \cup T) \cap \comp{A}] = A \cap \comp{A} \cap (B \cup T) = \emptyset \cap (B \cup T) = \emptyset$, and $B \cap \comp{X} = B \cap [(\comp{B} \cap \comp{T}) \cup A] = (B \cap \comp{B} \cap \comp{T}) \cup (B \cap A) = \emptyset \cap \emptyset = \emptyset$. Therefore $(A \cap X) \cup (B \cap \comp{X}) = \emptyset$ and so $X = (B \cup T) \cap \comp{A}$ is alternative form for solutions of the equation in part (d).}
\end{enumerate}

\item Show that for arbitrary sets $A, B, C, D$, and $X$,
\begin{enumerate}
\solution{$\comp{[(A \cap X) \cup (B \cap \comp{X})]} = (\comp{A} \cap X) \cup (\comp{B} \cap \comp{X})$}
{$\comp{[(A \cap X) \cup (B \cap \comp{X})]} = \comp{A \cap X} \cap \comp{B \cap \comp{X}} = (\comp{A} \cup \comp{X}) \cap (\comp{B} \cup X))$.\\Since $X$ is an arbitrary set we may reassign $X \leftarrow \comp{X}$.\\Then we have $\comp{[(A \cap X) \cup (B \cap \comp{X})]} = (\comp{A} \cup X) \cap (\comp{B} \cup \comp{X})$.}

\solution{$[(A \cap X) \cup (B \cap \comp{X})] \cup [(C \cap X) \cup (D \cap \comp{X})] = [(A \cup C) \cap X] \cup [(B \cup D) \cap \comp{X}]$}
{\begin{align*}[(A \cap X) \cup (B \cap \comp{X})] \cup [(C \cap X) \cup (D \cap \comp{X})] &= (A \cap X) \cup (B \cap \comp{X}) \cup (C \cap X) \cup (D \cap \comp{X})\\&= (A \cap X) \cup (C \cap X) \cup (B \cap \comp{X}) \cup (D \cap \comp{X})\\&= [(A \cup C) \cap X] \cup [(B \cup D) \cap \comp{X}]\end{align*}}

\solution{$[(A \cap X) \cup (B \cap \comp{X})] \cap [(C \cap X) \cup (D \cap \comp{X})] = [(A \cap C) \cap X] \cup [(B \cap D) \cap \comp{X}]$}
{\begin{align*}&[(A \cap X) \cup (B \cap \comp{X})] \cap [(C \cap X) \cup (D \cap \comp{X})]\\ &= \left[[(A \cap X) \cup (B \cap \comp{X})] \cap (C \cap X)\right] \cup \left[[(A \cap X) \cup (B \cap \comp{X})] \cap (D \cap \comp{X})\right]\\&= \left[[(A \cap X) \cap (C \cap X)] \cup [(B \cap \comp{X}) \cap (C \cap X)]\right]\\&\hspace{0.7cm}\cup \left[[(A \cap X) \cap (D \cap \comp{X})] \cup [(B \cap \comp{X}) \cap (D \cap \comp{X})]\right] \\&= (A \cap C \cap X) \cup (B \cap C \cap X \cap \comp{X}) \cup (A \cap D \cap X \cap \comp{X}) \cup (B \cap D \cap \comp{X})\\&= (A \cap C \cap X) \cup \emptyset \cup \emptyset \cup (B \cap D \cap \comp{X}) \\&= (A \cap C \cap X) \cup (B \cap D \cap \comp{X}) \\&= [(A \cap C) \cap X] \cup [(B \cap D) \cap \comp{X}] \end{align*}}
\end{enumerate}

\solution{Using the results in Exercises 5.7 and 5.8, prove that the equation $$(A \cap X) \cup (B \cap \comp{X}) = (C \cap X) \cup (D \cup \comp{X})$$ has a solution iff $B + D \subseteq \comp{A + C}$. In this event determine all solutions.}
{...}

\hrulefill


\end{enumerate}





































\end{document}
