\documentclass[12pt]{book}
\parindent=0px
\usepackage{amssymb, amsmath, fullpage, graphicx}
\usepackage{enumitem, siunitx}
\usepackage[utf8]{inputenc}
\usepackage[T1]{fontenc}

\newcommand{\set}[1]{\{#1\}}
\newcommand{\Set}{\text}

\newcommand{\defn}[1]{\textbf{#1}}
\newcommand{\solution}[2]{\item #1\\ \textbf{Solution}: #2}
\newcommand{\thm}[1]{\underline{\textsc{#1}}}

\newcommand{\qed}{\blacksquare}
\newcommand{\comp}{\overline}
\newcommand{\env}[2]{\begin{#1} #2 \end{#1}}
\newcommand{\tuple}[1]{\langle#1\rangle}

\newcommand{\then}{\Rightarrow}
\renewcommand{\iff}{\Leftrightarrow}
\newcommand{\goesto}{\rightarrow}

\newcommand{\floor}[1]{\lfloor#1\rfloor}
\newcommand{\ceil}[1]{\lceil#1\rceil}

% sets %
\newcommand{\NAT}{\mathbb{N}}
\newcommand{\INT}{\mathbb{Z}}
\newcommand{\RAT}{\mathbb{Q}}
\newcommand{\REAL}{\mathbb{R}}
\newcommand{\COMP}{\mathbb{C}}

% function families from function to boolean %
\newcommand{\Inj}[1]{\text{Inj}\left(#1\right)}
\newcommand{\Surj}[1]{\text{Surj}\left(#1\right)}
\newcommand{\Bij}[1]{\text{Bij}\left(#1\right)}



\begin{document}
\chapter{Sets and Relations}

\section{Cantor's Concept of a Set}
A \defn{set} $S$ is any collection of definite, distinguishable objects of our intuition or of our intellect to be conceived as a whole. The objects are called the \defn{elements} or \defn{members} of $S$.

\section{The Basis of Intuitive Set Theory}
\defn{Membership relation}: $x \in A$ if the object $x$ is a member of the set $A$. If $x$ is not a member of $A$ then $x \notin A$. $x_1, x_2, \dots, x_n \in A$ is shorthand for $x_1 \in A \wedge x_2 \in A \dots x_n \in A$.\\
\defn{The intuitive principle of extension}: Two sets are equal iff they have the same members.
\defn{Set equality}: The equality of two sets $X$ and $Y$ will be denoted by $X = Y$ and inequality of $X$ and $Y$ by $X \neq Y$. Among the basic properties of this relation are:
\begin{gather*}
X = X,\\X = Y \Rightarrow Y = X,\\ X = Y \wedge Y = Z \Rightarrow X = Z,
\end{gather*}
for all sets $X, Y$, and $Z$.\\
\defn{unit set}: a set $\set{x}$ whose sole member is $x$.\\
\defn{collection of sets}: a set whose members are sets.\\
\defn{The intuitive principle of abstraction}: A formula $P(x)$ defines a set $A$ by the convention that the members of $A$ are exactly those objects $a$ such that $P(a)$ is a true statement, denoted by $A = \set{x \mid P(x)}$.\\
Note: $\set{x \in A \mid P(x)} := \set{x \mid x \in A \wedge P(x)}$. For a property $P$ and function $f$ we can write $\set{f(x)\mid P(x)} := \set{y \mid \exists x\colon P(x) \wedge y = f(x)}$.

\section{Inclusion}
If $A$ and $B$ are sets, then $A$ is \defn{included in} $B$ iff each member of $A$ is a member of $B$. Symbolized: $A \subseteq B$. We also say that $A$ is a \defn{subset} of $B$. Equivalently, $B$ \defn{includes} $A$, symbolized by $B \supseteq A$.\\ The set $A$ is \defn{properly included in} $B$ ($A$ is a \defn{proper subset} of $B$ / $B$ \defn{properly includes} $A$) iff $A \subseteq B$ and $A \neq B$.\\ Among the basic properties of the inclusion relation are
\begin{gather*}
X \subseteq X;\\ X \subseteq Y \wedge Y \subseteq Z \Rightarrow X \subseteq Z;\\ X \subseteq Y \wedge Y \subseteq X \Rightarrow X = Y.
\end{gather*}
\defn{empty set}: $\set{x \in A \mid x \neq x}$ for any set $A$ is the set with no elements, symbolized by $\emptyset$.\\
\defn{power set}: the set of all subsets of a given set. $\mathcal{P}(A) = \set{B \mid B \subseteq A}$ for a given set $A$.

\section{Operations for Sets}
\defn{union}: for sets $A$ and $B$, the set of all objects which are members of either $A$ or $B$. $A \cup B = \set{x \mid x \in A \text{ or } x \in B}$. (\defn{sum}/\defn{join})\\
\defn{intersection}: for sets $A$ and $B$, the set of all objects which are members of both $A$ and $B$. $A \cap B = \set{x \mid x \in A \text{ and } x \in B}$. (\defn{product}/\defn{meet})\\
\defn{disjoint}: $A \cap B = \emptyset$ for sets $A$ and $B$.\\
\defn{intersect}: $A \cap B \neq \emptyset$ for sets $A$ and $B$.\\
\defn{disjoint collection}: for a collection of sets, each distinct pair of its member sets is disjoint.\\
\defn{partition}: for a set $X$, a disjoint collection $\mathcal{A}$ of nonempty and distinct subsets of $X$ such that each member of $X$ is a members of some (exactly one) member of $\mathcal{A}$.\\
\defn{absolute complement} of $A$: $\comp{A} = \set{x \mid x \notin A}$, the set of all members which are not in $A$.\\
\defn{relative complement} of $A$ with respect to $X$: $X - A = X \cap \comp{A} = \set{x \in X \mid x \notin A}$, the set of those members of $X$ which are not members of $A$.\\
\defn{symmetric difference} of $A$ and $B$: $A + B = (A - B) \cup (B - A)$.\\
\defn{universal set}: the set $U$ such that all sets under consideration in a certain discussion are subsets of $U$.\\

\end{document}